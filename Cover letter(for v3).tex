% LaTeX file for resume 
% This file uses the resume document class (res.cls)

\documentclass{article} 
% the margin option causes section titles to appear to the left of body text 
\textwidth=5.2in % increase textwidth to get smaller right margin
%\usepackage{helvetica} % uses helvetica postscript font (download helvetica.sty)
%\usepackage{newcent}   % uses new century schoolbook postscript font 
\usepackage{url}
\usepackage{hyperref}

\input{commands.tex}%

\begin{document} 
 
\title{Edits to ``Equivariant dendroidal Segal spaces and $G$-$\infty$-operads'' (version 3)
\\[12pt]} % the \\[12pt] adds a blank line after name
 
\author{Bonventre, P. and Pereira, L. A.}
 
\maketitle
 
 
\section{Requested revisions}
\begin{itemize}
\item[Section 5.2] Corrected typo, replacing $\otimes_{\Omega[C]}$ with $\coprod_{\Omega[C]}$. Added contextualization of this decomposition. 
\item[p.47-49 of Appendix A] Clarified the definition of $K \ltimes \mathbb R$ and extended discussion
\end{itemize}

\section{Typos and instances of slight rewording}
 
The following lists the corrections of typos listed in the preliminary referee report.

\begin{itemize}
\item[1.] added the names of the fibrant objects when naming the model structures in question, as well as included a version of the table produced by the reviewer (and references to the categorical narrative) for additonal clarity

\item[3.] added missing relation $aef \leq r$ to ``The other broad relations obtained by broad transitivity are \dots''
\item[7.] added ``$\varphi$''
\item[10.] reworded to ``denote the category of $G$-objects''
\item[16.] fixed ``of a leaf'' typo
\item[17.] changed ``does not act'' to ``acts trivially''
\item[18.] Deleted ``In fact''
\item[19.] Rephrased the $T/G$ remark to mention at the beginning that $T/G$ is formally defined below.
\item[20.] Replaced ``is an orbital $G$-subset'' with ``is a transitive $G$-subset''
\item[22.] added ``$\emptyset \neq E$''
\item[28.] explained what ``associative and unital'' means for the isomorphisms $U_i \xrightarrow{g} U_{gi}$
\item[29.] replaced instances of $Y$ with $B$
\item[31.] replaced ``subsets'' with ``simplicial subsets'' in the discussion of covers (``which in our language are the simplicial subsets'')
\item[32.] added comment noting that the ``dendroidal cover'' definition extends to the equivariant context
\item[33.] added the missing word ``planar'' to ``$G$-poset of outer faces''
\item[35.] added comment saying that in the proof of the second ``anodyne horn inclusion'' lemma we again assume $T \in \Omega^G \subseteq \Omega_G$.
\item[36.] reworded the definition of hypersaturated class
\item[38.] fixed $\Lambda^EF[T]$ typo
\item[45.] Changed ``Theorem 6.1'' to ``Thm. 6.1'' when citing MW09 (in the Remark discussing why strict lifting properties in $\mathsf{dSet}^G$ are unsatisfactory)
\item[46.] changed ``genuine'' and ``naive'' to ``fine'' and ``coarse''.
\item[47.] Added missing dot to ``Thm. 2.1'' in the citation from Per17 (at the beggining of \S 4)
\item[50.] added comment saying that ``$\square$ denotes the pushout product''
\item[51.] added $n \geq 1$ condition
\item[52.] added $m \geq 1$ condition
\item[54.] fixed ``monormorphism'' typo
\item[57.] added $n \geq 1$ condition
\item[59.] clarified that it is the ``simplicial structure maps $X_0 \to X_n$'' that need to be weak equivalences
\item[60.] replaced ``equivalences in $\mathsf{dSet}^G$'' by ``weak equivalences in $\mathsf{dSet}^G$'' in part (ii) of the Corollary
\item[61.] clarified that it is the ``natural maps'' that need to be Kan equivalences
\item[71.] added ``normal'' to ``a map $X\to Y$ in $\mathsf{PreOp}^G$ will have the right lifting property against all monomorphisms''
\item[73.] fixed $\mathsf{Preop}^G$ typo
\item[75.] fixed $x_n$ instead of $x_k$ typos
\item[77.] Recalled that $\mathsf{O}_G$ is the orbit category.

Similarly, a similar comment was added in the remark immediately preceding \S 4.
\item[79.] fixed $\varphi$ to be $\psi$
\item[80.] changed the $H$-equivalence from ``$f$'' to ``$j$'' to avoid confusion
\item[82.] fixed ``automorphim'' typo
\item[83.] removed the word ``degeneracy'' in the discussion of generalized Reedy category structure on $G \times \mathbb{S}$
\item[84.] added comment saying that ``$L_r$ and $M_r$ are recalled below'' before the theorem
\item[86.] changed ``genuine'' to ``fine''
\item[87.] Fixed the indexes in the ``extension of partial factorizations'' lemma
\end{itemize}




\section{Incorrect references}

\begin{itemize}
\item[6.] Fixed reference [BP17, Prop. 3.21] to [BP17, Prop. 3.23], and clarified the citation for BP17.
\item[48.] Fixed reference to ``Prop. 3.4.18'' of Hir03 to be ``Thm Prop. 3.3.18'' (I had an old version of the paper).

In addition, some extra detail concerning how \cite[Prop. 3.3.18]{Hir03} was being used was added, and some of the notation in Proposition 4.1 and its proof was adjusted to allow for this.

In further addition, made an observation as to why cofibrant approximation is a non issue.

\item[49.] Fixed reference to ``Thm. 3.3.8'' of Hir03 to be ``Thm 3.2.13'' (I had an old version of the paper)

\item[85.] Modified the reference \cite[\S 4]{BM11} to \cite[\S 4,\S 6]{BM11}, to note that \S 6 is where (co)skeleta are actually discussed. 
\end{itemize}



\section{Minor corrections}

\begin{itemize}
\item[8.] Clarified the definition of ``outer face map'' and corrected ``non-identity'' to be ``non-isomorphism''.
\item[41.] added a Remark to the end of \S 2.1 comparing $\Omega$ to $\mathsf{sSet}$, and introduced the stick tree $\eta$.
\item[56.] Added adjunction equations, highlighed second definition, 
\item[78.] Corrected and clarifed equation describing the coefficient system of categories part of the homotopy genuine operad.
\item fixed typo in the remark stating that $N[m] \to N \widetilde{[m]}$ is built from $0$-horns. Namely, ``$m$-simplex $\underline{a}$'' should have been ``$k$-simplex $\underline{a}$''
\end{itemize}

\section{Misc.}


\begin{itemize}
\item updated the reference to ``Equivariant Dendroidal Sets'' and other articles to reflect their publication

      Additionally, checked all references to the paper to make sure that they match the final version.
\item Added minor exposition in \S 5.1
\end{itemize}


\section{Departures from suggestions/direct answer to comments}

\begin{itemize}
\item[24.] clarified the class of cellular maps, and added references for nomenclature.
      % (in particular, distinguished it from the class of saturated maps)
      % Reworded ``is cellular on'' to ``is built by attaching'' $G$-inner horn inclusions
\item[25.] added a reference to Goodwillie for the use of the adjective ``convex'', and included the name ``downward closed. As our argument is similar in spirit to Goodwillie's, we kept to using his nomenclature

\item[27.] The new (Ch2) follows from the original conditions, but it requires both the original (Ch2) and (Ch3), which work in tandem. The discussion was expanded to provide more detail on this. 

As a side note, (Ch2) and (Ch3) are complementary, and if it weren't for the fact that equivariance forces us to worry about isotropies, I believe they could be combined without much worry.
However, separating (Ch3) is needed to control the isotropies (which is why (Ch3) is needed when verifying condition (c) in the proof). In fact, in earlier iterations of the characteristic edge lemma condition (Ch3) used to make explicit reference to elements $g \in G$, before the current version of (Ch0) was identified.

\item[30.] Corrected $A_C = 
A \cup \bigcup_{g\in G,V \in C} g b_{[e]}(\Omega[V])$.

Similarly, added the alternate $g \Omega[V]$ formula already in the proof of the characteristic edge lemma.

Moreover, expanded Remark FACEGACT REM to explain why $\Omega[gV] = g\Omega[V]$ and better explain the connection with isotropy of representable subpresheaves.



\item[37.] I do not think the proof of Proposition HYPER PROP should require anything analogous to the mentioned condition (b) in the proof of Lemma CHAREDGE LEM. The point is that for $U \in \Omega$ and $V \hookrightarrow$ a planar face, it is $|V|<|U|$ unless $V=U$. We've added an extra sentence to Remark FACCES REM that should make this point clearer.

\item[39.] The given argument for the Segal hypersaturation indeed used $T \in \Omega^G$. Since this made the discussion somewhat awkward, we've moved the argument being made to (the  new) Proposition SCANOD PROP, and rephrased accordingly.



\item[40.] this is a matter of taste, but I find changing $T$ to $S$ in $\Lambda^E_o[T] \to \Omega[T]$ to contrast with $Sc[T] \to \Omega[T]$ to be a bit confusing. To emphasize that both of the $T$ variables are free, rather than chosen, I've instead added ``$T\in \Omega_G$'' to both expressions.

\item[53.] The adjunctions from this notation are barely used in this part of the exposition. At most, part (i) of Proposition SSSETJREE PROP uses a single instance, which is also simpler since only the single $\partial \Delta[m]$ isn't a representable. It feels somewhat unnecessary to fully expand on the adjunctions at this point in the discussion. 

% not used in this part of the exposition, and in general the statements are cleaner and clearer when the two directions have different source categories.

% This is not quite true! The adjunction is necessary when proving part (i) of Proposition SSSETJREE PROP!

\item[65.] As written, Corollary REGGENHORN COR was indeed needed to finish the argument in Corollary JOINTFIBCHAR COR. This has been addressed by adding the generating horn inclusions $\Lambda^{Ge}[T] \to \Omega[T]$ to Proposition HYPER PROP.


\end{itemize}



\section{Important stuff I'm not sure how to qualify}


\begin{itemize}
\item[2.] 
added Remark RECOVDEF REM which explains why our results extend [CM13a, Thm. 6.6] and clarified clarified the influence of [CM13a, Thm. 6.6] in the introduction (to the paper and \S 4), 
%and extended the last remark of \S 5 to include an analogue of the precise statement in the equivariant setting
%\item[2.] clarified the influence of [CM13a, Thm. 6.6] in the introduction (to the paper and \S 4), and added a result after JOINTFIBCHAR COR to include a analogue of that statement in the equivariant setting

\item[9.] Rephrased the language in the ``$U^{\cup}$ and $U^{\cap}$'' proof to explain why the argument isn't circular. The main point being that ``edge'' and ``vertex'' are simply suggestive names for ``element'' and ``generating relation'' of the broad poset.


\item[11.] Converted the previous ``factorization of orbital faces'' remark to a proposition and proved it.

Edited Proposition UNIQUEFACT PROP to also include a statement about planar maps.

Expanded the notation defining ``outer closure'' to clarify both its meaning and why such an outer face exists.


\item[12.] Reworded the proposition to state that $GU$ is the smallest orbital face containing $U$.

Changed the language in the proof to reflect the restatement and provide some extra detail.


\item[13.] The ``$\leq_d$-incomparability of components'' language was removed in favor of a better explanation for the remark in point 15 below (which was moved to before this result)


\item[14.] Reworded the ending of the $GU$ proof to clarify the argument and be consistent with the rewording of the proposition itself.

\item[15.] The remark concerning when $G\cdot_H U \to T$ is injective was moved to an earlier point, and a clearer explanation was provided.

\item[21.] Provided references for the characterization of $G$-normal monomorphisms.
 
\item[23.] Adapted the definition of outer closure to also include non-planar faces.

\item[26.] Added a new paragraph explaining why conditions (a)(b)(c) imply that the given square is a pushout.

\item[34.] Adapted the definition of $GV$ to also include non-planar faces.

\item[42.] Clarified the meaning of hypersaturation when not assuming that the maps are cofibrations.

\item[43.] Clarified that the ``single colored $G$-operad'' is on the category of sets.

\item[44.] added comment that $\Omega(T)$ is the operad generated by $T$ and added references.

In addition, included the $\Omega(T)$ notation and references near the beginning of \S 2.1 when the similarity between the broad poset and the operad generated by $T$ is mentioned.

\item[55.] Fixed the math typo in the proof where $G$ should have been $\Lambda^i[n] \times G$.

In addition, a large portion of this proof has been rewritten with some slight changes in notation and more details provided.

\item[56.] Added a new remark discussing the two-variable adjunction and associated lifting properties.

\item[58.] A brief discussion of cofibrant generation for Reedy model structures was added to the end of the appendix. The proof of this result was reworded accordingly.

\item[62.] The confusing wording has been changed (the reason for the wording was the fact that I was trying to point out that the condition in the (new) HYPERMODEL REM had to be checked).

Added reference to HYPERMODEL REM in SSSETJREE PROP and in HYPERSIMPL REM

\item[63.] Edited proof to explain why the given condition shows $X_L,X_k $ are fibrant.

\item[64.] Edited the proof to explain why the maps $\Omega[T] \otimes (\{i\} \to J)$ suffice.

\item[66.] Edited the Notation defining $J^m$ to provide a little more information.

Added a new remark immediately after said notation proving that $J^{\bullet}$ is Reedy cofibrant cosimplicial.

Added reference to this new remark in the discussion in Remark CONCRECOM REM.

Added reference to this new remark in the proof of Proposition COMPLE PROP.

Added a definition of the bisimplicial $X(-)$ notation immediately before Proposition SESP PROP.

Edited Proposition SESP PROP and its proof to replace appearances of $n$ with $m$.

\item[67.] added Definition after the proof of the model structure on $\mathsf{PreOp}^G$, defining equivariant Segal operads,
and added Remark concerning the fibrant objects being the Reedy fibrant Segal operads.
%as fibrant objects
%%the Segal operads are not exactly the fibrant objects. It is convenient to have a larger class. This is used in the third paper once tame model structures enter the picture, since the tame fibrant stuff still satisfies the Segal operad condition, despite not being dendroidal Reedy fibrant

\item[68.] Explained the $\mathsf{csk}_{\eta}$ notation.

\item[69.] Reworded the proof describing the normal generators of $\mathsf{PreOp}^G$ to avoid the confusing wording.

\item[70.] Completed the proof describing the normal generators of $\mathsf{PreOp}^G$ by indicating why the generators are normal.

\item[72.] added explanation for why $X \otimes J$ is a cylinder object.

\item[74.] Added Proposition to beginning of \S 5.1 characterizing equivariant dendroidal Segal spaces in the vein of [CM13a, Cor. 5.6]

\item[76.] Turned the claim that $ho(X)$ is a genuine equivariant operad into a proposition, and gave a full proof.

%added the actual formulas for $X(A)$, since they get used here...

added an alternate formula for $\upsilon_{\**}$ after DSETG EQ.

added remark UPSPUSHMON REM noting that $\upsilon_{\**}$ preserves certain pushouts

%added internal references which show that $ho(X)$ is a genuine equivariant operad

\item[81.] added an explanation for why $X^h(J) = X(J)$.


\item expanded the discussion of cube fibrancy/cofibrancy before and during the proof of JDDK PROP, and (hopefully) made the overall argument a little clearer.

\item added reference for why $X^{\Omega[T]}$ is Segal space in Proposition JDDK PROP.

\item fixed the incorrect $\iota^{\**}_G$ notation issue by circumventing that notation.

\item Added short proof to the family model structure adjunction
 result.

\item Added a little more detail to the discussion of THM71 EX to better explain the role of elementary trees.

\end{itemize}




\end{document} 




%%% Local Variables:
%%% mode: latex
%%% TeX-master: t
%%% End:
