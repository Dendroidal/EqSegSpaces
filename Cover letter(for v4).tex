% LaTeX file for resume 
% This file uses the resume document class (res.cls)

\documentclass{article} 
% the margin option causes section titles to appear to the left of body text 
\textwidth=5.2in % increase textwidth to get smaller right margin
%\usepackage{helvetica} % uses helvetica postscript font (download helvetica.sty)
%\usepackage{newcent}   % uses new century schoolbook postscript font 
\usepackage{url}
\usepackage{hyperref}
\hypersetup{
  % colorlinks,
  final,
  pdftitle={Equivariant Dendroidal Segal Spaces},
  pdfauthor={Bonventre, P. and Pereira, L. A.},
  % pdfsubject={Your subject here},
  % pdfkeywords={keyword1, keyword2},
  linktoc=page
}

\usepackage{xr}
\externaldocument{EqSegSp&G-infty-ops}

\input{commands.tex}%

%-------- TIKZ -----------------------------------------
\usepackage{tikz}%
\usetikzlibrary{matrix,arrows,decorations.pathmorphing,
cd,patterns,calc}
\tikzset{%
  treenode/.style = {shape=rectangle, rounded corners,%
                     draw, align=center,%
                     top color=white, bottom color=blue!20},%
  root/.style     = {treenode, font=\Large, bottom color=red!30},%
  env/.style      = {treenode, font=\ttfamily\normalsize},%
  dummy/.style    = {circle,draw,inner sep=0pt,minimum size=2mm}%
}%

\usetikzlibrary[decorations.pathreplacing]
% \usetikzlibrary{external}\tikzexternalize
% \makeatletters
% \renewcommand{\todo}[2][]{\tikzexternaldisable\@todo[#1]{#2}\tikzexternalenable}

% \makeatother

\begin{document} 
 
\title{Edits to ``Equivariant dendroidal Segal spaces and $G$-$\infty$-operads'' (v.4)
\\[12pt]} % the \\[12pt] adds a blank line after name
 
\author{Bonventre, P. and Pereira, L. A.}
 
\maketitle

The referee's second report listed 14 points that needed revision,
all of which we believe have been addressed, as detailed below.








\section{Outstanding issues from previous report}

% The two locations which caused the referee confusion have been been updated.

\begin{itemize}
	\item[4.] In the introductory discussion to \S 2,
	a footnote was added clarifying that the term 
	\emph{maximum} as used in \cite{Per18}
	is meant to refer to a ``greatest element'' or 
	an ``unique maximal element''.
	
	Given the referee's clarification on the issue, 
	it seems that we agree on the correct definition of node/root,
	and that the confusion is actually one of nomenclature.
	The convention I (Luis) was taught is that the term \emph{maximum}
	is reserved for an unique maximal element
	(so that ``maximum'' and ``maximal element'' aren't synonyms),
	though in retrospect this may be because the expression 
	``greatest element'' (which is more common in English) doesn't translate well to Portuguese.
	
	On a side note, Weiss' original draft \cite{Wei12}
	uses the same convention, 
	as does the Wikipedia page on ``Maximal and minimal elements'',
	so my usage is not unprecedented in English.
	
	We would be happy to add further clarification
	on this point, should that seem necessary.
		
	\item[25.] All instances of ``convex subset'' (and ``convexity'') were suitably replaced with ``lower set''.
	
	Personally, I (Luis) was averse to this change,
	since I used the term ``convex'' in the
	``Equivariant Dendroidal Sets'' prequel \cite{Per18},
	and changing terminology mid project seems like bad practice,
	but if this clashes with commonly used terminology, I'm fine with changing it.
	
	Though I feel obliged to note a peculiar irony of this situation:
	In a draft to a previous paper I used the term
	``downward closed'' to refer to such subsets,
	and was then asked by the referee to 
	instead follow Goodwillie and use the term ``convex'',
	which is why I followed that nomenclature in 
	``Equivariant Dendroidal Sets''.
\end{itemize}





\section{Requested appendix changes}

\begin{itemize}
	
\item[1.] Revised the beginning of Notation \ref{KLTIMES NOT}
to allow $K$ to be any group acting on $\mathbb{R}$,
not just $K \leq \mathsf{Aut}_{\mathsf{Cat}}(\mathbb{R})$. 	

\item[2.] Fixed $\{\mathcal{F}_r\}$
to $\mathcal{F}_r$ in Lemma \ref{REEDYTRCOF LEM}(ii)
     
\item[3.] Replaced ``analogue'' with ``analogous'' at the beginning of in the proof of Lemma \ref{REEDYTRCOF LEM}(ii)

\item[4.]
In the last line of the proof of Lemma \ref{REEDYTRFIB LEM}(ii),
replaced ``(i) again...'' with ``Part (i) again...''
so as not to start sentence with ``(i)''.

\item[5.]
At the end of Remark \ref{ITERREEDY REM}
fixed $\mathcal{C}$
to be $\mathcal{C}^{\mathbb{R} \times \mathbb{S}}$,
so that $f\colon A \to B$ is a map of diagrams
\end{itemize}








\section{Requested changes in \S \ref{REZKCOMP SEC}}
 
% The following lists the corrections of typos and other smaller mistakes listed in the preliminary referee report.

\begin{itemize}
	\item[1.]
	Revised the wording in the last paragraph of the proof of 
	Proposition \ref{COMPLE PROP}
	to make explicit reference to the conditions (i),(ii)
	in the statement
	\item[2.]
	The paragraph in the proof of Proposition \ref{COMPLE PROP}
	explaining why $\widetilde{X}$
	is complete Segal was revised to make the argument and terminology clear.
\end{itemize}



\section{Requested changes elsewhere}

% The following lists the corrections of typos and other smaller mistakes listed in the preliminary referee report.

\begin{itemize}
	\item[1.] In the proof of Prop. \ref{HYPER PROP} clarified that the reference is to the proof of Prop. \ref{SCANOD PROP}, rather than Prop. \ref{SCANOD PROP} itself
	\item[2.] In Notation \ref{JM NOT}
	replaced ``$J^m$ will always...'' with
	``The term $J^m$ will always...''
	to avoid starting the sentence with a symbol
	\item[3.] Replaced ``localizing the maps'' with ``localizing with respect to the maps'' in Remark 4.27
	\item[4.] Fixed ``$X(T)$'' to be ``$X(\Omega[T])$''
	in Definition 4.40
	\item[5.] Moved the remark characterizing the fibrant objects in
	$\mathsf{PreOp}^G$ from the end of \S 4 to the end of \S 5,
	where it is now Corollary \ref{SEGALOP_COR}.
\end{itemize}


\section{Other changes}

% The following lists the corrections of typos and other smaller mistakes listed in the preliminary referee report.

\begin{itemize}
	\item Added a little more detail to the proof of 
	Proposition \ref{SESP PROP}(ii).
	The previous wording might have been misinterpreted as saying that the spine $\mathsf{Spi}[m] = \Delta[1] \vee_{\Delta[0]} \cdots \vee_{\Delta[0]} \Delta[1]$
	coincided with the $1$-skeleton $\mathsf{sk}_1\Delta[m]$
	\item Slightly rephrased the last paragraph 
	(concerning essential surjectivity)
	of the proof of Proposition \ref{JDDK PROP}
	to clarify the argument
	and fix a typo
	(the map
	$X(\{0\} \times J) \to X (J \times J)$
	was written backwards). 
\end{itemize}



\bibliography{biblio}{}

\bibliographystyle{alpha}


\end{document} 




%%% Local Variables:
%%% mode: latex
%%% TeX-master: t
%%% End:
