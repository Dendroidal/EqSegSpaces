% LaTeX file for resume 
% This file uses the resume document class (res.cls)

\documentclass{article} 
% the margin option causes section titles to appear to the left of body text 
\textwidth=5.2in % increase textwidth to get smaller right margin
%\usepackage{helvetica} % uses helvetica postscript font (download helvetica.sty)
%\usepackage{newcent}   % uses new century schoolbook postscript font 
\usepackage{url}
\usepackage{hyperref}
\hypersetup{
  % colorlinks,
  final,
  pdftitle={Equivariant Dendroidal Segal Spaces},
  pdfauthor={Bonventre, P. and Pereira, L. A.},
  % pdfsubject={Your subject here},
  % pdfkeywords={keyword1, keyword2},
  linktoc=page
}

\usepackage{xr}
\externaldocument{EqSegSp&G-infty-ops}

\input{commands.tex}%

%-------- TIKZ -----------------------------------------
\usepackage{tikz}%
\usetikzlibrary{matrix,arrows,decorations.pathmorphing,
cd,patterns,calc}
\tikzset{%
  treenode/.style = {shape=rectangle, rounded corners,%
                     draw, align=center,%
                     top color=white, bottom color=blue!20},%
  root/.style     = {treenode, font=\Large, bottom color=red!30},%
  env/.style      = {treenode, font=\ttfamily\normalsize},%
  dummy/.style    = {circle,draw,inner sep=0pt,minimum size=2mm}%
}%

\usetikzlibrary[decorations.pathreplacing]
% \usetikzlibrary{external}\tikzexternalize
% \makeatletters
% \renewcommand{\todo}[2][]{\tikzexternaldisable\@todo[#1]{#2}\tikzexternalenable}

% \makeatother

\begin{document} 
 
\title{Edits to ``Equivariant dendroidal Segal spaces and $G$-$\infty$-operads'' (v.4)
\\[12pt]} % the \\[12pt] adds a blank line after name
 
\author{Bonventre, P. and Pereira, L. A.}
 
\maketitle
 
The referee's preliminary report had two main sections:
first, they highlighted two key points where unclear notation prevented further reading of those sections; % of great narrative difficulty,
and second, they offered a numbered list of recommendations and suggestions.
We have implemented the vast majority of the changes recommended by the referee, and modified the discussion at many points to minimize confusion and provide additional detail.
The updates are organized into four categories:
\begin{itemize}
\item in ``Requested Revisions'', we address the two key pieces of unclear notation.
\item in ``Comments and departures'', we answer questions posed by the referee, and indicate where and why we departed from the referee's suggestions.
\item in ``Changes'', we address suggestions which required more significant edits,
% to the article, or small changes regarding errors which had inhibited mathematical understanding. These changes have themselves been grouped thematically.
and which we group thematically.
\item in ``Minor changes and typos'', we list fixed typos and other small corrections. % made in response to errors which did not appear to affect the referee's understanding.
\end{itemize}

Our updates are numbered by the corresponding numbered item in the referee report,
and all internal references have been modified to match the new numbering from the updated article.






\section{Outstanding issues from previous report}

% The two locations which caused the referee confusion have been been updated.

\begin{itemize}
	\item[4.] In the introductory discussion to \S 2,
	a footnote was added clarifying that the term 
	\emph{maximum} as used in \cite{Per18}
	is meant to refer to a ``greatest element'' or 
	an ``unique maximal element''.
	
	Given the referee's clarification on the issue, 
	it seems that we agree on the correct definition of node/root,
	and that the confusion is merely one of nomenclature.
	The convention I (Luis) was taught is that the term \emph{maximum}
	is reserved for an unique maximal element,
	though in retrospect this may be because the expression 
	``greatest element'' (which seems more common in English) doesn't translate well to Portuguese.
	
	On a side note, Weiss' original draft \cite{Wei12}
	uses the same convention, 
	as does the Wikipedia page on ``Maximal and minimal elements'',
	so my usage is not unprecedented in English.
	
	Please let us now if additional clarification should be added to the draft.
	
	\item[25.] All instances of ``convex subset'' (and ``convexity'') were suitably replaced with ``lower set''.
	
	Personally, I (Luis) was a bit averse to this change,
	since I have already used the term ``convex'' in the
	``Equivariant Dendroidal Sets'' prequel \cite{Per18},
	and changing terminology mid project seems like bad practice,
	but if this clashes with terminology that is indeed commonly used, I'm fine with changing it.
	
	Though I feel obliged to note a rather peculiar irony of this situation:
	In a draft to a previous paper I used the term
	``downward closed'' to refer to such subsets,
	and was then asked by the referee to 
	instead follow Goodwillie and use the term ``convex'',
	which is why I followed that nomenclature in 
	``Equivariant Dendroidal Sets''.
\end{itemize}




{\color{red} HERE}

\section{Minor appendix changes}

\begin{itemize}
	
\item[1.] Revised the beginning of Notation \ref{KLTIMES NOT}
to allow $K$ to be any group acting on $\mathbb{R}$,
not just $K \leq \mathsf{Aut}_{\mathsf{Cat}}(\mathbb{R})$. 	

\item[2.] Fixed $\{\mathcal{F}_r\}$
to $\mathcal{F}_r$ in Lemma \ref{REEDYTRCOF LEM}(ii)
     
\item[3.] Replaced ``analogue'' with ``analogous'' in the proof of Lemma \ref{REEDYTRCOF LEM}(ii)

\item[4.]
In the last line of the proof of Lemma \ref{REEDYTRFIB LEM}(ii),
replaced ``(i) again...'' with ``Part (i) again...''
so as not to start sentence with ``(i)''.

\item[5.]
At the end of Remark \ref{ITERREEDY REM}
fixed $\mathcal{C}$
to be $\mathcal{C}^{\mathbb{R} \times \mathbb{S}}$,
so that $f\colon A \to B$ is a map of diagrams
\end{itemize}








\section{Minor changes in \S \ref{REZKCOMP SEC}}
 
% The following lists the corrections of typos and other smaller mistakes listed in the preliminary referee report.

\begin{itemize}
	\item[1.]
	Revised the wording at the end of the proof of 
	Proposition \ref{COMPLE PROP}
	to make explicit reference to the conditions (i),(ii)
	in the statement
	\item[2.]
	The paragraph in the proof of Proposition \ref{COMPLE PROP}
	explaining why $\widetilde{X}$
	is complete Segal was revised to make the argument and terminology clear.
\end{itemize}



\section{Minor changes suggested by the referee elsewhere}

% The following lists the corrections of typos and other smaller mistakes listed in the preliminary referee report.

\begin{itemize}
	\item[1.] In the proof of Prop. \ref{HYPER PROP} clarified that the reference is to the proof of Prop. \ref{SCANOD PROP}, rather than Prop. \ref{SCANOD PROP} itself
	\item[2.] In Notation \ref{JM NOT}
	replaced ``$J^m$ will always...'' with
	``The term $J^m$ will always...''
	to avoid starting the sentence with a symbol
	\item[3.] Replaced ``localizing the maps'' with ``localizing with respect to the maps'' in Remark 4.27
	\item[4.] Fixed ``$X(T)$'' to be ``$X(\Omega[T])$''
	in Definition 4.40
	\item[5.] Moved the remark characterizing the fibrant objects in
	$\mathsf{PreOp}^G$ from the end of \S 4 to the end of \S 5,
	where it is now Corollary \ref{SEGALOP_COR}.
\end{itemize}


\section{Other changes}

% The following lists the corrections of typos and other smaller mistakes listed in the preliminary referee report.

\begin{itemize}
	\item Slightly rephrased the last paragraph 
	(concerning essential surjectivity)
	of the proof of Proposition \ref{JDDK PROP}
	to clarify the argument
	and fix a typo
	(the map
	$X(\{0\} \times J) \to X (J \times J)$
	was written backwards). 
\end{itemize}



\bibliography{biblio}{}

\bibliographystyle{alpha}


\end{document} 




%%% Local Variables:
%%% mode: latex
%%% TeX-master: t
%%% End:
