
%%%%%%%%%%%%%%%%%% PACKAGES %%%%%%%%%%%%%%%%%%%%%%%%%%%%

\usepackage{amsmath, amsfonts, amssymb, amsthm}%

% ------------- No Idea What These Do ---------------------

\usepackage[all,cmtip]{xy}%
\newdir{ >}{{}*!/-5pt/@{>}}%
\usepackage{array}%
\usepackage[dvipsnames]{xcolor}%
\usepackage[normalem]{ulem}%
\usepackage{dsfont}%
\usepackage{bbm}%
\usepackage{paralist}%

\usepackage{relsize}%

% ------ Margins/Rotating -----------------------------------

\usepackage{geometry}%[margin=1in]

% ------ New Characters --------------------------------------

\usepackage[latin1]{inputenc}%
\usepackage{MnSymbol}
\usepackage{stmaryrd}
\usepackage{upgreek}
\usepackage{mathrsfs}
% \usepackage[T1]{fontenc}
% \usepackage[english]{babel}
% \usepackage{fouriernc}


%----- Enumerate ---------------------------------------------
\usepackage{enumerate}%
\usepackage[inline]{enumitem}%



%%%%%%%%%%%%%%%%%%%%%%%%% INTERNAL REFERENCES %%%%%%%%%%%%%%%%%%%%%%%%%%%%%%%%%%%
\usepackage{mathtools}
\mathtoolsset{showonlyrefs}



\usepackage[capitalise]{cleveref}  % load after hyperref
\newcommand{\clevertheorem}[3]{%
  \newtheorem{#1}[theorem]{#2}
  \crefname{#1}{#2}{#3}
}



% ------- New Theorems/ Definition/ Names-----------------------

\theoremstyle{plain} % bold name, italic text
\newtheorem{theorem}{Theorem}[section]
\crefname{theorem}{Theorem}{Theorems}
\newtheorem*{theorem*}{Theorem}
\clevertheorem{proposition}{Proposition}{Propositions}
\clevertheorem{lemma}{Lemma}{Lemmas}
\clevertheorem{corollary}{Corollary}{Corollaries}
\clevertheorem{conjecture}{Conjecture}{Conjectures}
\newtheorem*{conjecture*}{Conjecture}%
\clevertheorem{claim}{Claim}{Claims}
% \newtheorem{theorem}[equation]{Theorem}%
% \newtheorem*{theorem*}{Theorem}%
% \newtheorem{lemma}[equation]{Lemma}%
% \newtheorem{proposition}[equation]{Proposition}%
% \newtheorem{corollary}[equation]{Corollary}%
% \newtheorem{conjecture}[equation]{Conjecture}%
% \newtheorem*{conjecture*}{Conjecture}%
% \newtheorem{claim}[equation]{Claim}%

%%%%%% Fancy Numbering for Theorems
% \newtheorem{innercustomgeneric}{\customgenericname}
% \providecommand{\customgenericname}{}
% \newcommand{\newcustomtheorem}[2]{%
%   \newenvironment{#1}[1]
%   {%
%    \renewcommand\customgenericname{#2}%
%    \renewcommand\theinnercustomgeneric{##1}%
%    \innercustomgeneric
%   }
%   {\endinnercustomgeneric}
% }

% \newcustomtheorem{customthm}{Theorem}
% \newcustomtheorem{customcor}{Corollary}
%%%%%%%%%%%%%

\theoremstyle{definition} % bold name, plain text
\clevertheorem{definition}{Definition}{Definitions}
\newtheorem*{definition*}{Definition}
\clevertheorem{example}{Example}{Examples}
\clevertheorem{remark}{Remark}{Remarks}
\clevertheorem{notation}{Notation}{Notations}
\clevertheorem{convention}{Convention}{Conventsion}
\clevertheorem{assumption}{Assumption}{Assumptions}
\clevertheorem{exercise}{Exercise}{Exercises}
% \newtheorem{definition}[equation]{Definition}%
% \newtheorem*{definition*}{Definition}%
% \newtheorem{example}[equation]{Example}%
% \newtheorem{remark}[equation]{Remark}%
% \newtheorem{notation}[equation]{Notation}%
% \newtheorem{convention}[equation]{Convention}%
% \newtheorem{assumption}[equation]{Assumption}%
% \newtheorem{exercise}{Exercise}%


\makeatletter\let\c@equation\c@theorem\makeatother %% This makes equations follow the theorem counter
\makeatletter\let\c@figure\c@theorem\makeatother %% This makes figures follow the theorem counter

\crefname{figure}{Figure}{Figures}



% % -------Table of Contents FOR AMSART ------------------------------
% \makeatletter
% % Table of Contents
% \setcounter{tocdepth}{3}

% % Add bold to \section titles in ToC and remove . after numbers
% \renewcommand{\tocsection}[3]{%
%   \indentlabel{\@ifnotempty{#2}{\bfseries\ignorespaces#1 #2\quad}}\bfseries#3}
% % Remove . after numbers in \subsection
% \renewcommand{\tocsubsection}[3]{%
%   \indentlabel{\@ifnotempty{#2}{\ignorespaces#1 #2\quad}}#3}
% % \let\tocsubsubsection\tocsubsection% Update for \subsubsection
% % ...

% \newcommand\@dotsep{4.5}
% \def\@tocline#1#2#3#4#5#6#7{\relax
%   \ifnum #1>\c@tocdepth % then omit
%   \else
%   \par \addpenalty\@secpenalty\addvspace{#2}%
%   \begingroup \hyphenpenalty\@M
%   \@ifempty{#4}{%
%     \@tempdima\csname r@tocindent\number#1\endcsname\relax
%   }{%
%     \@tempdima#4\relax
%   }%
%   \parindent\z@ \leftskip#3\relax \advance\leftskip\@tempdima\relax
%   \rightskip\@pnumwidth plus1em \parfillskip-\@pnumwidth
%   #5\leavevmode\hskip-\@tempdima{#6}\nobreak
%   \leaders\hbox{$\m@th\mkern \@dotsep mu\hbox{.}\mkern \@dotsep mu$}\hfill
%   \nobreak
%   \hbox to\@pnumwidth{\@tocpagenum{\ifnum#1=1\bfseries\fi#7}}\par% <-- \bfseries for \section page
%   \nobreak
%   \endgroup
%   \fi}
% \AtBeginDocument{%
%   \expandafter\renewcommand\csname r@tocindent0\endcsname{0pt}
% }
% \def\l@subsection{\@tocline{2}{0pt}{2.5pc}{5pc}{}}
% \makeatother



%%%%%%%%%%%%%%%%%%%%%%%%%%%%%%%%%%%%%%%%%%%%%%%%%%%%%%%%%%%%%%%%%%%%%%%%%
% --------- Shortcut Commands ------------------------------

\newcommand{\A}{\mathcal{A}}%
\newcommand{\B}{\mathcal{B}}%
\newcommand{\G}{\mathcal{G}}%
\newcommand{\V}{\ensuremath{\mathcal V}}
\newcommand{\s}{\mathbf{s}}%
\renewcommand{\t}{\mathbf{t}}%
\renewcommand{\H}{\mathcal{H}}%
\newcommand{\Ss}{\mathbb S}%
\newcommand{\Tt}{\mathbb T}%
\newcommand{\Rr}{\mathbb R}%
\newcommand{\Qq}{\mathbb Q}%
\newcommand{\Cc}{\mathbb C}%
\newcommand{\Zz}{\mathbb Z}%


\newcommand{\m}{\text{$\mathfrak{m}$}}
\newcommand{\norm}[1]{\left\Vert#1\right\Vert}%
\newcommand{\abs}[1]{\left\vert#1\right\vert}%
\newcommand{\set}[1]{\left\{#1\right\}}%
\newcommand{\sets}[2]{\left\{ #1 \;|\; #2\right\}}%
\newcommand{\eval}[1]{\left\langle#1\right\rangle}%
\newcommand{\eps}{\varepsilon}%
\newcommand{\To}{\longrightarrow}%
\newcommand{\rmap}{\longrightarrow}%
\newcommand{\longto}{\longrightarrow}%
\newcommand{\lmap}{\longleftarrow}%
\newcommand{\into}{\hookrightarrow}%
\newcommand{\onto}{\twoheadrightarrow}%
\newcommand{\Boxe}{\raisebox{.8ex}{\framebox}}%
\newcommand{\X}{\mathcal{X}}%
\newcommand{\F}{\ensuremath{\mathbb{F}}}%
\newcommand{\E}{\ensuremath{\mathcal{E}}}%
\newcommand{\D}{\ensuremath{\mathcal{D}}}%
\newcommand{\C}{\ensuremath{\mathcal{C}}}%
\renewcommand{\P}{\ensuremath{\mathcal{P}}}%
\renewcommand{\O}{\ensuremath{\mathcal{O}}}%
%\renewcommand{\S}{\ensuremath{\mathcal{S}}}%
\renewcommand{\Ss}{\ensuremath{\mathsf{S}}}%
\newcommand{\su}{\ensuremath{\mathfrak{su}}}%
\newcommand{\Pa}{\ensuremath{\mathcal{M}}}%
\newcommand{\NN}{\ensuremath{\mathcal{N}}}%
\newcommand{\AAA}{\ensuremath{\mathcal{A}}}%
\newcommand{\Z}{\ensuremath{\mathcal{Z}}}%
\newcommand{\mcI}{\ensuremath{\mathcal{I}}}%
\newcommand{\al}{\alpha}%
\newcommand{\be}{\beta}%
\newcommand{\ga}{\gamma}%
\newcommand{\ksi}{\xi}
\newcommand{\Ksi}{\Xi}
\newcommand{\Lie}{\mathcal{L}}%  
\renewcommand{\gg}{\mathfrak{g}}%
\newcommand{\hh}{\mathfrak{h}}%
\renewcommand{\L}{\mathbb L}%
\newcommand{\tto}{\rightrightarrows}%

\newcommand{\del}{\partial}%
\newcommand{\fC}{\ensuremath{\mathfrak{C}}}%
\newcommand{\fc}{\ensuremath{\mathfrak{c}}}%
\newcommand{\blank}{\,\line(1,0){5}\,}%


\newcommand{\Sig}{\ensuremath{\Sigma}}%

\newcommand{\Sym}{\ensuremath{\mathsf{Sym}}}%
\newcommand{\Fin}{\mathsf{F}}%
\newcommand{\Set}{\ensuremath{\mathsf{Set}}}
\newcommand{\Top}{\ensuremath{\mathsf{Top}}}
\newcommand{\sSet}{\ensuremath{\mathsf{sSet}}}%
\newcommand{\Op}{\ensuremath{\mathsf{Op}}}%
\newcommand{\sOp}{\ensuremath{\mathsf{sOp}}}%
\newcommand{\BSym}{\ensuremath{\mathsf{BSym}}}%
\newcommand{\fgt}{\ensuremath{\mathsf{fgt}}}%

\DeclareMathOperator{\hocmp}{hocmp}%
\DeclareMathOperator{\cmp}{cmp}%
\DeclareMathOperator{\hofiber}{hofiber}%
\DeclareMathOperator{\fiber}{fiber}%
\DeclareMathOperator{\hocofiber}{hocof}%
\DeclareMathOperator{\hocof}{hocof}%
\DeclareMathOperator{\holim}{holim}%
\DeclareMathOperator{\hocolim}{hocolim}%
\DeclareMathOperator{\colim}{colim}%
\DeclareMathOperator{\Lan}{Lan}%
\DeclareMathOperator{\Map}{Map}%
\DeclareMathOperator{\Id}{Id}%
\DeclareMathOperator{\mlf}{mlf}%
\DeclareMathOperator{\Hom}{Hom}%
\DeclareMathOperator{\Aut}{Aut}%
\newcommand{\free}{\ensuremath{\mathrm{free}}}%
\newcommand{\Stab}{\ensuremath{\mathrm{Stab}}}%




