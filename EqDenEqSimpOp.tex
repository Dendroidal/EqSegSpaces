\documentclass[a4paper,10pt]{article}%

\usepackage[hidelinks]{hyperref}
\hypersetup{
%  colorlinks,
  linktoc=page
}
\usepackage{bookmark}


\input{commands.tex}%

\usepackage{showkeys}

\author{Peter Bonventre, Lu\'is A. Pereira}%
\title{Equivariant dendroidal sets and simplicial operads}%

\usepackage{stmaryrd}

\usepackage{upgreek}
\usepackage{mathtools}

\usepackage{geometry}

\usepackage{tikz}%
\tikzset{%
  treenode/.style = {shape=rectangle, rounded corners,%
                     draw, align=center,%
                     top color=white, bottom color=blue!20},%
  root/.style     = {treenode, font=\Large, bottom color=red!30},%
  env/.style      = {treenode, font=\ttfamily\normalsize},%
  dummy/.style    = {circle,draw,inner sep=0pt,minimum size=2mm}%
}%

\usetikzlibrary[decorations.pathreplacing]
% \usetikzlibrary{external}\tikzexternalize
% \makeatletter
% \renewcommand{\todo}[2][]{\tikzexternaldisable\@todo[#1]{#2}\tikzexternalenable}
% \makeatother

\begin{document}	\maketitle%



\abstract{bla bla, generalizing \cite{CM13a}.

Bla, also obtain an equivariant notion of Reedy category}


\tableofcontents

\section{Equivariant dendroidal sets}




\subsection{Preliminaries}

\begin{definition}
      A map $f: S_0 \to T_0$ in $\Omega$ is called a \textit{face map} if it is injective on underlying sets.
      A face map is \textit{inner} if it is of the form $T_0 \setminus E \into T_0$, where $E$ is a subset of the set of inner edges of $T$.      
      Fixing such a subset $E$, let $\Phi_{\mathrm{Inn}}^E(T_0)$ denote the poset (under inclusion) of
      all inner face maps $S_0 \into T_0$
      such that $E \subseteq T_0 \setminus S_0$.
\end{definition}

\begin{definition}
      Given $S_0 \in \Omega$, $T \in \Omega_G$, and a map of forests $f: S_0 \to T$, let
      $T_0$ denote the component of $T$ containing the image of $S_0$.

      We say $f$ is a (non-equivariant) \textit{(inner) face map} if
      $f: S_0 \to T_0$ is a non-equivariant (inner) face map.

      Given a subseteq $E$ of inner edges, let
      $\Phi_{\mathrm{Inn}}^E(T)$ denote the subposet of faces such that miss all of $E$.
\end{definition}

\begin{definition}
      Fix $T\in \Omega_G$, and a component $T_0$ of $T$, with $H := \mathrm{Stab}_G(T_0)$.
      
      Let $\Phi(T)$ denote the poset (under inclusion) of face maps whose image is
      strictly containined in $T_0$.
      Given an inner edge $e \in T$, let
      $\Phi^{Ge}(T) = \Phi(T) \setminus (T_0 \setminus H e)$.
      Define the \textit{$Ge$-horn} of $T$ to be the subdendroidal set
      \begin{equation}
            \Lambda^{Ge}[T] := \mathop{\colim}\limits_{\Phi^{Ge}(T)}\Omega[G \cdot S_0].
      \end{equation}
\end{definition}

\begin{definition}
      A face map $f: S_0 \to T$ is called \textit{orbital} if
      $f(S_0) \subseteq T$ is $K$-closed, where $K = \mathrm{Stab}_G(f(r_s))$, for $r_S$ the root of $S_0$.

      Fixing a component $T_0$ of $T$,
      let $\Phi_{o}(T)$ denote the poset (under inclusion) of orbital face maps whose image is
      strictly contained in $T_0$.
      Define the \textit{orbital boundary} of $T$ to be the subdendroidal set
      \begin{equation}
            \partial_{o}\Omega[T] := \colim\limits_{\Phi_{\mathrm{Orb}}(T)}\Omega[G \cdot S_0].
      \end{equation}

      Given an inner edge $e \in T$, let
      $\Phi_{\mathrm{Orb}}^{Ge}(T) := \Phi_{o}(T) \setminus (T_0\setminus H e)$
      where $H = \mathrm{Stab}_G(T_0)$
      (that is; those faces $S_0$ such that either $T \setminus (G.f(S_0) \cup Ge) \neq \varnothing$, or
      outer faces removing a stump).
      Define the \textit{$Ge$-orbital horn} to be the subdendroidal set
      \begin{equation}
            \Lambda^{Ge}_{o}[T] := \colim\limits_{\Phi_{o}^{Ge}(T)}\Omega[G\cdot S_0]. 
      \end{equation}
\end{definition}

\begin{definition}
      Given $T$, $T_0$, and $H$ as above,
      let $U_0 \subseteq T_0$ be a (non-equivariant) subtree, and
      let $K = \mathrm{Stab}_H(U_0)$.
      Suppose we have another subdendroidal set $X \subseteq \Omega[T]$ which contains all outer faces of $U$, and
      an edge $e \in U_0$.
      We say that $K e \subseteq U_0$ is a \textit{characteristic edge orbit} of
      $X \subseteq \Omega[T] \supseteq \Omega[G \cdot U_0]$
      if we have
      \begin{equation}
            e\in R_0 \in \Phi_{\mathrm{Inn}}(U) \cap X \mbox{ if and only if } R_0/ \bar K e \in \Phi_{\mathrm{Inn}}(U) \cap X,
      \end{equation}
      where $\bar K = \mathrm{Stab}_{K}(R_0)$. 
\end{definition}

\begin{proposition}
      \label{CHAR_EDGE_ANODYNE_PROP}
      Given $T$, $T_0$, $H$, $U_0$, and $X$ as above, we have that
      $K e$ is a characteristic edge orbit of $X \subseteq \Omega[T] \supseteq \Omega[G \cdot U]$
      implies that $X \to \X \cup \Omega[G \cdot U]$ is inner $G$-anodyne.
\end{proposition}
\begin{proof}
      If $\Omega[U]$ is already contained in $X$, we are done.
      Assuming otherwise, it suffices to show that for all
      $C \subseteq C'$ $K$-closed concave subsets of $\Phi_{\mathrm{Inn}}^{K e}(U_0) \setminus X$, the map
      \begin{equation}
            X \cup G \cdot_K \left( \bigcup\limits_{E \in C}\Omega[U_0 \setminus E] \right)
            \to
            X \cup G \cdot_K \left( \bigcup\limits_{E' \in C'}\Omega[U_0 \setminus E'] \right)                     
      \end{equation}
      is inner $G$-anodyne.
      Indeed, once $C = \Phi_{\mathrm{Inn}}^{K e}(U_0) \setminus X$, we have the pushout
      \begin{equation}
            \begin{tikzcd}
                  G \cdot_K \left( \Lambda^{K e}[U_0] \right) \arrow[d] \arrow[r]
                  &
                  X \cup G \cdot_K \left( \bigcup\limits_{E \in \Phi_{\mathrm{Inn}}^{K e}(U_0)} \Omega[U_0 \setminus E] \arrow[d] \right)
                  \\
                  G \cdot_K \Omega[U_0] \simeq \Omega[G \cdot U_0] \arrow[r]
                  &
                  X \cup \Omega[G \cdot U_0].
            \end{tikzcd}
      \end{equation}
      Moreover, it suffices to consider $C' = C \cup H.D$ for $D \subseteq U_0$, where
      without loss of generality $e \in D$ and $U_0 \setminus D$ is not in the domain.
      Let $\bar K = \mathrm{Stab}_H(D)$.

      We first claim that $\Lambda^{K e}[U_0 \setminus D]$ is in the domain.
      If $S_0$ is an outer face of $U_0 \setminus D$, then
      $S_0$ factors through an outer face of $U_0$, and so $S_0$ is in $X$.
      Further, if $S_0 = U_0 \setminus (D \cup E)$ with $E \cap K e = \varnothing$,
      then concavity implies that $S_0$ is in the domain, as required.

      Second, we claim that no face $S_0 = U_0 \setminus D \cup \bar e$, with $\bar e \subseteq K e$, is in the domain.
      Suppose $U_0 \setminus D \cup \bar e$ is contained in some $U_0 \setminus E$ already attached.
      Then, since $E \cap K e = \varnothing$< we have $U_0 \setminus D \subseteq U_0 \setminus E$, so
      $U_0 \setminus D$ is also in the domain, a contradiction.
      Further, if $U_0 \setminus D \cup \bar e$ is in $X$, then so is $U_0 \setminus D \cup K e$, and hence
      so is $U_0 \setminus D$ (by definition of characteristic edge orbit), also a contradiction.

      Now, all faces $U_0 \setminus D \cup \bar e$ with $\varnothing \subseteq \bar e \subseteq K e$ have stabilizer $\bar K$
      (else $D \cap K e \neq \varnothing$).
      Thus the desired map is the pushout of
      \begin{equation}
            G \cdot_{\bar K} \left( \Lambda^{\bar K e}[U_0 \setminus D] \to \Omega[U_0 \setminus D] \right),
      \end{equation}
      and hence is anodyne, as required.
\end{proof}






\todo[inline]{come back}


\begin{lemma}
      Suppose $U_0$ is a minimal outer face of $T$ not in $\Lambda_0^{Ge}[T]$, and suppose $e\in U_0$.
      Then $K e$ is a characteristic edge orbit for $\Lambda_0^{Ge}[T] \subseteq \Omega[T] \supseteq[G \cdot U_0]$.
\end{lemma}
\begin{proof}
      Consider an inner face $U_0 \setminus D$ of $U_0$, with $\bar K = \mathrm{Stab}_H(U_0 \setminus D) \leq K$, and
      suppose $U_0 \setminus D \cup \bar K e \in \Lambda_0^{Ge}[T]$.
      Then
      \begin{equation}
            K_r = \mathrm{Stab}_H(r_U) \leq \mathrm{Stab}_H(U_0 \setminus D \cup \bar K e) \leq \mathrm{Stab}_H(U_0) \leq Kr,
      \end{equation}
      so the outer face $U_0$ is $K_r$-closed and not in $\Lambda_0^{Ge}[T]$, and hence we conclude that $U_0$ must be $T_0$.
      However, if $\Lambda^{Ge}_0[T] \neq \Lambda^{Ge}[T]$, $T_0$ is not a minimal outer face not in $\Lambda_0^{Ge}[T]$.
      Thus, in these cases, $K e$ is a characteristic edge orbit vacuously.
      If in fact $\Lambda_o^{Ge}[T] = \Lambda^{Ge}[T]$, then $T \simeq G \cdot T_0$, so $H = \bar K = K_r = \set{e}$.
      Now, since $U_0 \setminus D \cup e \in \Lambda_0^{Ge}[T]$,
      $D \setminus e \neq \varnothing$, so
      $U_0 \setminus D \in \Lambda_0^{Ge}[T]$.
\end{proof}

\begin{proposition}
      $\Lambda_o^{Ge}[T] \to \Omega[T]$ is inner $G$-anodyne, and hence
      the (hyper)saturated class of the orbital horn inclusions is contained in
      the (hyper)saturated class of the horn inclusions.
\end{proposition}
\begin{proof}
      Let $\mathrm{Out}_o(T)$ be the poset of outer faces $U_0$ of $T$ which are not in $\Lambda_0^{Ge}[T]$.
      It suffices to show that for any $G$-closed convex subsets $B \subseteq B'$ of $\mathrm{Out}_o(T)$, the map
      \begin{equation}
            \Lambda_o^{Ge}[T] \cup \mathop{\bigcup}\limits_{R_0 \in B}\Omega[G \cdot R_0]
            \to
            \Lambda_o^{Ge}[T] \cup \mathop{\bigcup}\limits_{R_0' \in B'}\Omega[G \cdot R_0']
      \end{equation}
      is inner $G$-anodyne.
      Again, suffices to show when $B' = B \cup \set{U_0}$. Let $K = \mathrm{Stab}_G(U_0)$, and
      without loss of generality assume $e \in U_0$.
      The case $B = \varnothing$ is the previous lemma.
      For general $B$, we know the domain contains all outer faces of $U_0$ by convexity.
      Now, let $U_0 \setminus D$ be an inner face, with stabilizer $\bar K$.
      Then $U_0 \setminus D \cup \bar K e$ is in the domain if either
      $U_0 \setminus D \cup \bar K e$ is contained in some $\Omega[G \cdot R_0]$, or
      $U_0 \setminus D \cup \bar K e \in \Lambda_o^{Ge}[T]$.
      But then $U_0$ is contained in $R_0$ since both are outer faces, or again we apply the previous lemma.
      Both lead to contradictions, and thus we may conclude that
      $K e$ is a characteristic edge orbit for the domain relative to $\Omega[T] \supseteq \Omega[G \cdot U_0]$.
      Thus, the result holds by Proposition \ref{CHAR_EDGE_ANODYNE_PROP}.
\end{proof}

\subsection{Actual Stuff}


\begin{notation}
Given subgroups $H_i \leq G$, $0\leq i \leq k$ such that
$H_0 \geq H_i$, $1 \leq i \leq k$ we write
$C_{\amalg_i H_0/H_i}$ for the $G$-corolla encoding the 
$H_0$-set $H_0/H_1 \amalg \cdots \amalg H_0/H_k$.
\end{notation}


Following the discussion preceding \cite[Prop. 3.6.8]{HHM16}, we will call a class of maps of $\mathsf{dSet}^G$ \textit{hypersaturated} if is closed under pushouts, transfinite composition, retracts, and satisfies the following cancellation property: if in
\[
A \xrightarrow{f} B \xrightarrow{g} C
\]
both $f$ and $gf$ are in the class, then so is $g$.

The following is an equivariant generalization of 
\cite[Props. 2.4 and 2.5]{CM13a}.

\begin{proposition}\label{HYPER PROP}
The following sets of maps generate the same hypersaturated class:
\begin{itemize}
\item the $G$-inner horn inclusions
$\Lambda^{Ge} [T] \to \Omega[T]$ for $T \in \Omega_G$ and $Ge$ an inner edge orbit; 
\item the $G$-inner orbital horn inclusions
$\Lambda^{Ge}_o [T] \to \Omega[T]$ for $T \in \Omega_G$ and $Ge$ an inner edge orbit; 
\item the $G$-segal core inclusions
$Sc [T] \to \Omega[T]$ for $T \in \Omega_G$.
\end{itemize}

\end{proposition}

{\color{red} HERE}

\begin{remark}
	Setting $G=e$ and slicing over the stick tree $\eta$ in the previous result
	one recovers the more well known claim that 
	the hypersaturation (in fact, saturation) of the simplicial inner horns
	$\{\Lambda^i[n] \to \Delta[n] \colon 0< i < n\}$
	coincides with the hypersaturation of the simplicial Segal core inclusions
	$\{Sc[n] \to \Delta[n]\}$.
\end{remark}

\begin{remark}\label{HYPERSATKAN REM}
	We will also make use of a variant of the previous remark for the hypersaturation of \textit{all} simplicial horns.
	Namely, we claim that the hypersaturation of all simplicial horns 
	$\{\Lambda^i[n] \to \Delta[n] \colon 0 \leq i \leq n\}$
	coincides with the hypersaturation of all vertex inclusion maps
	$\{\Delta[0] \to \Delta[n]\}$.
	Indeed, call the latter hypersaturation $S$. 
	An easy argument shows that the Segal core inclusions 
	$\{Sc[n] \to \Delta[n]\}$ are in $S$ and thus so are all inner horn inclusions. On the other hand, the skeletal filtration of the left horns $\Lambda^0[n]$ is built exclusively out of left horn inclusions, and thus since $\Delta[0]=\Lambda^0[1] \to \Delta[1]$ is in $S$ so are all left horn inclusions 
	$\Lambda^0[n] \to \Delta[n]$. The case of right horn inclusions $\Lambda^n[n] \to \Delta[n]$ is dual.
\end{remark}


The following is the equivariant generalization of 
\cite[Thm. 3.5]{CM13a}.

\begin{proposition}\label{TFAE PROP}
Let $X \to Y$ be a map between $G$-$\infty$-operads. The following are equivalent:
\begin{enumerate}
	\item[(a)] for all $G$-corollas $C_A$ and $H\leq G$ the maps
\[k(\Omega[C_A],X) \to k(\Omega[C_A],Y), \qquad
k(\Omega[G/H \cdot \eta],X) \to k(\Omega[G/H \cdot \eta],Y)
\]
are weak equivalences in $\mathsf{sSet}$;
	\item[(b)] for all $G$-trees $T$ the maps 
	\[k(\Omega[T],X) \to k(\Omega[T],Y) \]
are weak equivalences in $\mathsf{sSet}$;
	\item[(c)] for all normal $G$-dendroidal sets $A$, the maps
	\[k(A,X) \to k(A,Y) \]
are weak equivalences in $\mathsf{sSet}$;
	\item[(d)] $f \colon X \to Y$ is a weak equivalence in 
	$\mathsf{dSet}^G$.
\end{enumerate}
\end{proposition}


\begin{definition}
	Let $X$ be a $G$-$\infty$-operad.
	A \textit{$G$-profile} on $X$ is a map
\[
	\partial \Omega[C] \to X
\]
	for some $G$-corolla $C \in \Sigma_G$.

More explicitly, a $G$-profile consists of:
\begin{itemize}
	\item subgroups $H_i \leq G$, $0\leq i \leq k$ such that
	$H_0 \geq H_i$ for $1 \leq i \leq k$;
	\item objects $x_i \in X(\eta)^{H_i}$ for $0 \leq i \leq k$.
\end{itemize}
To simplify notation, we will prefer to denote a $G$-profile as 
$(x_1,\cdots,x_k;x_0)$, and refer to it as a 
\textit{$C$-profile}.
\end{definition}


\begin{definition}
Given a $G$-$\infty$-operad and a $C$-profile 
$(x_1,\cdots,x_k;x_0)$ we define the space of maps
$X(x_1,\cdots,x_k;x_0)$ to be given by the pullback
\[
\begin{tikzcd}
	X(x_1,\cdots,x_k;x_0) \ar{r} \ar{d}&
	Hom(\Omega[C],X) \ar{d}
\\
	G \cdot \eta \ar{r}[swap]{(x_1,\cdots,x_k;x_0)} &
	\prod_{0\leq i \leq k} X(\eta)^{H_i}
\end{tikzcd}
\]
Noting that there are equivalences of categories (the first of which is an isomorphism)
\[(\mathsf{dSet}_G) / G\cdot \eta \simeq 
\mathsf{sSet}^{B_G} \simeq \mathsf{sSet},\]
 one sees that 
$X(x_1,\cdots,x_k;x_0)$ 
can indeed be regarded as a simplicial set (in fact, this is a Kan complex).
\end{definition}


\begin{definition}
Let $f \colon X \to Y$ be a map of $G$-$\infty$-operads.

The map $f$ is called \textit{fully faithful} if, for each $C$-profile $(x_1,\cdots, x_k ; x_0)$ one has that
\[
X(x_1,\cdots,x_k;x_0) \to Y\left(f(x_1),\cdots,f(x_k);f(x_0)\right)
\]
is weak equivalence in $\mathsf{sSet}$.

The map $f$ is called \textit{essentially surjective} if for each subgroup $H \leq G$ the map of categories
$\tau(\iota^{\**}(X^H)) \to \tau(\iota^{\**}(Y^H))$
are essentially surjective.
\end{definition}

The following is the equivariant generalization of 
\cite[Thm. 3.11 and Remark 3.12]{CM13a}.


\begin{theorem}
A map $f \colon X \to Y$ of $G$-$\infty$-operads is fully faithful iff for all $G$-corollas $C \in \Sigma_G$ the commutative squares
of Kan complexes
\begin{equation}\label{COMSQ EQ}
\begin{tikzcd}
	k(\Omega[C],X) \ar{r} \ar{d}[swap]{p}&
	k(\Omega[C],Y) \ar{d}{q}
\\
	k(\partial \Omega[C],X) \ar{r}[swap]{f_{\**}} &
	k(\partial \Omega[C],Y)
\end{tikzcd}
\end{equation}
are homotopy pullback squares.

Hence, $f$ is a weak equivalence in $\mathsf{dSet}^G$ iff $f$ is both fully faithful and essentially surjective. 
\end{theorem}

\begin{proof}
Noting that the $0$-simplices of $k(\partial \Omega[C],X)$
are precisely the $C$-profiles $(x_1,\cdots,x_k,x_0)$,
fully faithfulness can be reinterpreted as saying that all fiber maps
$p^{-1}(x_1,\cdots,x_k,x_0) \to 
q^{-1}(f(x_1),\cdots,f(x_k),f(x_0))$
are weak equivalences. But since $p,q$ are Kan fibrations, this is equivalent to the condition that (\ref{COMSQ EQ})
is a homotopy pullback (see \cite[Lemma 3.9]{CM13a}), and the first half follows.

For the second half, note first that the bottom map in 
(\ref{COMSQ EQ}) can be rewritten as
\[
	\prod_{0\leq i \leq k} k \left(G/H_i \cdot \eta, X \right) \to 
	\prod_{0\leq i \leq k} k \left(G/H_i \cdot \eta, Y \right).
\]

Assume first that $f$ is a weak equivalence. Proposition \ref{TFAE PROP} then implies that the horizontal maps in (\ref{COMSQ EQ})
are weak equivalences, so that the square is a pull back square, and thus $f$ is fully faithful. 
That $f$ is essentially surjective follows from the identity
$k \left(G/H \cdot \eta, X \right) = k(\iota^{\**}(Z^H))$, so that 
$\tau(\iota^{\**}(X^H)) \to \tau(\iota^{\**}(Y^H))$ is essentially surjective at the level of maximal groupoids, and this suffices for essential surjectivity.

Assume now that $f$ is fully faithful and essentially surjective. Since (\ref{COMSQ EQ}), Proposition \ref{TFAE PROP} now implies that one needs only check that the maps of Kan complexes
\begin{equation}\label{KANMAP EQ}
	k \left(G/H \cdot \eta, X \right) \to 
	k \left(G/H \cdot \eta, Y \right)
\qquad \text{or} \qquad
	k\left(\iota^{\**}\left(X^H\right)\right) \to 
	k\left(\iota^{\**}\left(Y^H\right)\right)
\end{equation}
are weak equivalences. As before, essential surjectivity is equivalent to the fact that the maps (\ref{KANMAP EQ}) induce surjections on connected components. Hence, it now suffices to show that for each $0$-simplex $x \in X^H$ the top map of loop spaces in 
\begin{equation}\label{OMEGASQ EQ}
\begin{tikzcd}
	\Omega(k(\iota^{\**}X^H),x) \ar{r} \ar{d} &
	\Omega(k(\iota^{\**}Y^H),f(x)) \ar{d}
\\
	X(x;x) \ar{r} &
	Y(f(x);f(x))
\end{tikzcd}
\end{equation}
is a weak equivalence. Noting that the bottom map in 
(\ref{OMEGASQ EQ})
is a weak equivalence since $F$ is fully faithful
 and that the vertical maps are the inclusion of the connected components corresponding to automorphisms of $x$ in 
 $\tau(\iota^{\**} X^H)$.
 It thus suffices to check that the top map in
 (\ref{OMEGASQ EQ}) is an isomorphism on $\pi_0$,
 and this follows since the map of categories
 $\tau(\iota^{\**}(X^H)) \to \tau(\iota^{\**}(Y^H))$
 is fully faithful. 
\end{proof}


\section{Equivariant simplicial dendroidal sets}

The results in \cite[\S 4]{CM13a}
concerning the simplicial Reedy model structure all generalized mutatis mutandis.

\begin{proposition}\label{COMBMODSTR PROP}
	Suppose $\mathcal{C}$
	admits two model structures $(C,W_1,F_1)$ and $(C,W_2,F_2)$
	with the same class of cofibrations, and assume further that both model structures are cofibrantly generated and admit left Bousfield localizations with respect to any set of maps.
	
	Then there is a smallest common left Bousfield localization 
	$(C,W,F)$ and for any $(C,W,F)$-local 
	$c,d$ objects one has that $c\to d$ is in $W$ iff it is in $W_1$ iff it is in $W_2$.
	
	Moreover, an object $X$ is local in the common left Bousfield localization iff it is simultaneously fibrant in both of the two initial model structures.
\end{proposition}

\begin{proof}
	The model structure $(C,W,F)$ can be obtained by either localizing $(C,W_1,F_1)$ with regards to the generating trivial cofibrations of $(C,W_2,F_2)$ or vice-versa. That the two processes yield the same model structure follows from from the universal property of left Bousfield localizations \cite[Prop. 3.4.18]{Hi03}.
	The claim concerning local $c,d$ follows from the local Whitehead theorem \cite[Thm. 3.3.8]{Hi03}, stating that
the local equivalences between local objects match the
initial weak equivalences.

That local objects are fibrant in both model structures follows since $C \cap W$ contains both $C \cap W_1$ and $C\cap W_2$ (in fact, this shows that local fibrations are fibrations in both model structures). The converse claim follows from the observation that fibrant objects in any model structure are local with respect to the weak equivalences in that same model structure.
\end{proof}

The prototypical example of Proposition \ref{COMBMODSTR PROP}
is given by the category $\mathsf{ssSet}$ of bisimplicial sets together with the two possible Reedy structures (over the Kan model structure on $\mathsf{sSet}$).

Explicitly, writing the levels of 
$X \in \mathsf{ssSet}$ as $X_{n,m}$
one can either form a Reedy model structure with respect to the 
\textit{horizontal index $n$}
or with respect to the 
\textit{vertical index $m$}.
In either case, the generating cofibrations are then given by the maps
\[
	\left( \partial \Delta[n] \to \Delta[n] \right)
\square
	\left( \partial \Delta[m] \to \Delta[m] \right),
\qquad n,m\geq 0.
\] 
Further, in the horizontal Reedy model structure the generating trivial cofibrations are the maps
\[
	\left( \partial \Delta[n] \to \Delta[n] \right)
\square
	\left( \Lambda^j[m] \to \Delta[m] \right),
\qquad n\geq 0,m\geq j \geq 0.
\]
while for the vertical Reedy model structure the generating trivial cofibrations are the maps
\[
	\left( \Lambda^i[n] \to \Delta[n] \right)
\square
	\left( \partial \Delta[m] \to \Delta[m] \right),
\qquad n \geq i \geq 0,m\geq 0.
\]
We caution the reader about a possible hiccup with the terminology: 
the weak equivalences for the horizontal Reedy structure are the 
\textit{vertical equivalences},
i.e. maps inducing Kan equivalences of simplicial sets
$X_{n,\bullet} \to Y_{n,\bullet}$
for each $n \geq 0$, and dually for the vertical Reedy structure.

In the next result we refer to the localized model structure given by Proposition \ref{COMBMODSTR PROP} as the 
\textit{joint Reedy model structure}.

\begin{proposition}\label{SSSETJREE PROP}
	Suppose that $X, Y \in \mathsf{ssSet}$ are horizontal Reedy fibrant. Then:
\begin{itemize}
	\item[(i)] for each fixed $m$ all vertex maps $X_{\bullet,m} \to X_{\bullet,0}$ are trivial Kan fibrations;
	\item[(ii)] any vertical Reedy fibrant replacement $\tilde{X}$ of $X$ is in fact fibrant in the joint Reedy model structure;
	\item[(iii)] a map $X \to Y$ is a horizontal weak equivalence iff it is a joint weak equivalence;
	\item[(iv)] the canonical map $X_{n,0} \to X_{n,n}$ is a Kan equivalence.
\end{itemize}
\end{proposition}

\begin{proof}
(i) follows since the trivial cofibrations for the horizontal Reedy structure include all the maps of the form
$(\partial \Delta[n] \to \Delta[n]) \square (\Delta[0] \to \Delta[m])$.

(ii) follows since by (i) $\tilde{X}$ is then local
(with the vertical Reedy model structure as the initial model structure)
with respect to all maps of the form
$\Delta[0] \times (\Delta[0] \to \Delta[m])$,
and thus by Remark \ref{HYPERSATKAN REM}
it is fibrant in the joint Reedy model structure ({\color{blue} add a remark about this}).

(iii) follows from (ii) since the localizing maps 
$X \to \tilde{X}$, $Y \to \tilde{Y}$
are horizontal equivalences.

For (iv), note first that the diagonal functor
$\Delta \colon \mathsf{ssSet} \to \mathsf{sSet}$
is left Quillen for either the horizontal or vertical Reedy structures (and thus also for the joint Reedy structure). But noting that all objects are cofibrant, and regarding 
$X_{n,0}$ as a bisimplicial set that is vertically constant, the claim
follows by noting that by (i) the map
$X_{n_0} \to X$ is a horizontal weak equivalence in $\mathsf{ssSet}$.
\end{proof}

\begin{corollary}
	A map $f\colon X \to Y$ in $\mathsf{ssSet}$ is a joint equivalence iff it induces a Kan equivalence on diagonals
	$\Delta(X) \to \Delta(Y)$ in $\mathsf{sSet}$.
\end{corollary}

\begin{proof}
	Since horizontal Reedy fibrant replacement maps
	$X \to \tilde{X}$ are diagonal equivalences, 
	one reduces to the case of $X,Y$ horizontal Reedy fibrant.
	
	But Proposition \ref{SSSETJREE PROP} (i) and (iii) then combine to say that $X \to Y$ is a joint equivalence iff
	$X_{\bullet,0} \to Y_{\bullet,0}$ is a Kan equivalence, 
	so that the result follows from Proposition \ref{SSSETJREE PROP} (iv).
\end{proof}


We now turn to our main application of Proposition \ref{COMBMODSTR PROP}, the category 
$\mathsf{sdSet}^G = \mathsf{Set}^{\Delta^{op} \times \Omega^{op} \times G}$
of $G$-equivariant simplicial dendroidal sets.

Using the fact that $\Delta$ is a (usual) Reedy category
and the model structure on $\mathsf{dSet}^G$ given by 
\cite[Thm. 2.1]{Pe17}
yields a model structure on $\mathsf{sdSet}^G$
that we will refer to as the \textit{simplicial Reedy model structure}.

On the other hand, in the context of Definition \ref{GENRED DEF},
$\Omega^{op} \times G$ is a generalized Reedy category such that the families $\{\mathcal{F}_{U}^{\Gamma}\}_{U \in \Omega}$
of $G$-graph subgroups are Reedy-admissible 
(see Example \ref{GGRAPHREEDY EX}), 
and hence using the underlying 
Kan model structure on $\mathsf{sSet}$, 
Theorem \ref{REEDYADM THM} yields
a model structure on $\mathsf{sdSet}^G$
that we will refer to as the \textit{equivariant dendroidal Reedy model structure}, 
or simply as \textit{dendroidal Reedy model structure} for the sake of brevity.

\begin{proposition}
	Both the simplicial and dendroidal Reedy model structures on 
	$\mathsf{sdSet}^G$ have generating cofibrations given by the maps
\[
	\left(\partial \Delta [n] \to \Delta[n]\right)
		\square
	\left(\partial \Omega[T] \to \Omega[T]\right),
	\qquad
	n\geq 0, T \in \Omega_G.
\]
Further, the dendroidal Reedy structure has generating trivial cofibrations the maps
\begin{equation}\label{DENDTRIVCOF EQ}
	\left(\Lambda^i [n] \to \Delta[n]\right)
		\square
	\left(\partial \Omega[T] \to \Omega[T]\right),
	\qquad
	n\geq i \geq 0, T \in \Omega_G.
\end{equation}
while the simplicial Reedy structure has generating trivial cofibrations the maps
\begin{equation}\label{SIMPTRIVCOF EQ}
	\left(\partial \Delta [n] \to \Delta[n]\right)
		\square
	\left(A \to B\right),
	\qquad
	n\geq 0
\end{equation}
for $\{A \to B\}$ a set of generating trivial cofibrations of
$\mathsf{dSet}^G$.
\end{proposition}

\begin{proof}
	For the claims concerning the dendroidal Reedy structure, 
	note that the presheaves $\Omega[T] \in \mathsf{dSet}^G$
	are precisely the quotients $(G \cdot \Omega[U])/K$ for $U\in \Omega$ and $K \leq G \times \Sigma_U$ a $G$-graph subgroup,
	so that $\partial \Omega[T] \to \Omega[T]$
	represents the maps $X_U^K \to (M_U X)^K$ for $X \in \mathsf{dSet}^G$.
	
	The claims concerning the simplicial Reedy structure are immediate.
\end{proof}


\begin{corollary}\label{JOINTFIBCHAR COR}
The joint fibrant objects $X \in \mathsf{sdSet}^G$ have the following equivalent characterizations:
\begin{itemize}
	\item[(i)] $X$ is both simplicial Reedy fibrant and dendroidal Reedy fibrant;
	\item[(ii)] $X$ is simplicial Reedy fibrant and all maps 
	$X_0 \to X_n$ are equivalences in $\mathsf{dSet}^{G}$;
	\item[(iii)] $X$ is dendroidal Reedy fibrant and all maps
\[
	X^{\Omega[T]} \to X^{Sc[T]}
\qquad \text{and} \qquad
	X^{\Omega[T]} \to X^{\Omega[T]\otimes J_d}
\]
for $T \in \Omega_G$ are Kan equivalences in $\mathsf{sSet}$.
\end{itemize}
\end{corollary}


\begin{proof}
	(i) simply repeats the last part of Proposition \ref{COMBMODSTR PROP}. In the remainder we write $K \to L$ for a generic monomorphism in 
$\mathsf{sSet}$
and $A \to B$ a generic normal monomorphism in $\mathsf{dSet}^G$.

	For (ii), note first that $X$ is simplicial fibrant iff 
$X^L \to X^K$ is always a fibration in $\mathsf{dSet}^G$. 
Hence, such $X$ will have the right lifting property againt all maps in (\ref{DENDTRIVCOF EQ}) iff 
$X^L \to X^K$ is a trivial fibration whenever $K \to L$ is anodyne. But Remark \ref{HYPERSATKAN REM}
implies that it suffices to verify this for
the vertex inclusions $\Delta[0] \to \Delta[n]$.

For (iii), note first that $X$ is dendroidal fibrant iff $X^B \to X^A$ is always a Kan fibration in $\mathsf{sSet}$.
Therefore, $X$ will have the right lifting property against all maps (\ref{SIMPTRIVCOF EQ}) iff 
$X^B \to X^A$ is a trivial Kan fibration whenever $A\to B$ is a generating trivial of $\mathsf{dSet}^G$.
By adjunction, this is equivalent to showing that
$X^L \to X^K$ is a fibration in $\mathsf{dSet}^G$ for any monomorphism $K \to L$ in $\mathsf{sSet}$. Moreover, by the fibration between fibrant objects part of \cite[Prop. 8.8]{Pe17}
(see also the beginning of \cite[\S 8.1]{Pe17})
it suffices to verify that the maps $X^L \to X^K$ have the right lifting property against the maps
\[
	\Lambda^{G e} \Omega[T] \to \Omega[T],
	\quad
	T \in \Omega_G, e \in Inn(T)
\qquad
\text{and}
\qquad
	\Omega[T] \otimes \left( \{i\} \to J_d\right),
	\quad
	T \in \Omega_G, i = \{0,1\}
\]
and it thus suffices to check that $X^B \to X^A$ is a trivial Kan fibration whenever $A\to B$ is one of these maps.
Proposition \ref{HYPER PROP} now finishes the proof.
\end{proof}

We now obtain the following partial analogue of Proposition \ref{SSSETJREE PROP}. Note that the equivalences in the simplicial Reedy model structure are the dendroidal equivalences and vice versa.

\begin{corollary}
	Suppose that $X, Y \in \mathsf{sdSet}^G$ are dendroidal Reedy fibrant. Then:
\begin{itemize}
	\item[(i)] for each fixed $m$ all vertex maps $X_{\bullet,m} \to X_{\bullet,0}$ are trivial fibrations in $\mathsf{dSet}^G$;
	\item[(ii)] any simplicial Reedy fibrant replacement $\tilde{X}$ of $X$ is in fact fibrant in the joint Reedy model structure;
	\item[(iii)] a map $X \to Y$ is a dendroidal weak equivalence iff it is a joint weak equivalence;
	\item[(iv)] regarding $X_0$ as a simplicially constant object in $\mathsf{sdSet}^G$, the map $X_0 \to X$ is a dendroidal equivalence, and thus a joint equivalence. (iv) follows from (i).
\end{itemize}
\end{corollary}

\begin{proof}
	The proof adapts that of Proposition \ref{SSSETJREE PROP}.
	(i) follows since $X$ then has the right lifting property with respect to all maps 
	$(\Delta[0] \to \Delta[m]) \square (\partial \Omega[T] \to \Omega[T])$. (ii) follows from (i) and the characterization in  Corollary \ref{JOINTFIBCHAR COR} (ii). (iii) follows from (ii) since the simplicial fibrant replacement maps 
	$X \to \tilde{X}$ are dendroidal equivalences.
\end{proof}


\begin{theorem}
	The inclusion/$0$-th level adjunction
	\[
	\iota\colon 
	\mathsf{dSet}^G \rightleftarrows \mathsf{sdSet}^G
	\colon (-)_0,
	\]
	where $\mathsf{sdSet}^G$ is given the joint Reedy model structure,
	is a Quillen equivalence.
\end{theorem}

\begin{proof}
	It is clear that the inclusion preserves both normal monomorphisms and all weak equivalences, hence the adjunction is Quillen. 
	Consider any map $\iota(A) \to X$ with $X$ joint fibrant and perform a trivial cofibration followed by fibration factorization on the left
	\[\iota(A) \xrightarrow{\sim} \widetilde{\iota(A)} \to X
		\qquad
		A \xrightarrow{\sim} \widetilde{\iota(A)}_0 \to X_0
	\]
	for the simplicial Reedy model structure. 
	Corollary \ref{JOINTFIBCHAR COR} (ii) now implies that 
	$\widetilde{\iota(A)}$ is in fact joint fibrant
	and thus that the leftmost composite above is a joint equivalence iff $\widetilde{\iota(A)} \to X$ is a dendroidal equivalence in $\mathsf{sdSet}^G$ iff $\widetilde{\iota(A)}_0 \to X_0$ is an equivalence in  $\mathsf{dSet}^G$ iff the rightmost composite is an equivalence in $\mathsf{dSet}^G$.
\end{proof}



{\color{red} HERE HERE}

\section{Pre-operads}

kl


\newpage

\appendix

\section{Equivariant Reedy model structures}

In \cite{BM08} Berger and Moerdijk extend the notion of Reedy category so as to allow for categories $\mathbb{R}$
 with non-trivial automorphism groups 
 $\mathsf{Aut}(r)$ for $r \in \mathbb{R}$.
For such $\mathbb{R}$ and suitable model category $\mathcal{C}$ they then show that there is a 
\textit{Reedy model structure}
on $\mathcal{C}^{\mathbb{R}}$
that is defined by modifying the usual characterizations of
Reedy cofibrations, weak equivalences and fibrations
(see \cite[Thm. 1.6]{BM08} or
Theorem \ref{REEDYADM THM} below)
 to be determined by the $\mathsf{Aut}(r)$-projective model structures
on $\mathcal{C}^{\mathsf{Aut}(r)}$
for each $r \in \mathbb{R}$. 

The purpose of this appendix is to show that,
under suitable conditions, this can also be done by replacing
the $\mathsf{Aut}(r)$-projective model structures
on $\mathcal{C}^{\mathsf{Aut}(r)}$
with the more general 
$\mathcal{C}^{\mathsf{Aut}(r)}_{\mathcal{F}_r}$
model structures for 
$\{\mathcal{F}_r\}_{r \in \mathbb{R}}$
a nice collection of families of subgroups of each 
$\mathsf{Aut}(r)$.

To do so, we first need some essential notation.
For each map $r \to r'$ in a category $\mathbb{R}$ we will write
$\mathsf{Aut}(r \to r')$ for its automorphim group in the arrow category and write
\begin{equation}\label{PIDEFR EQ}
\begin{tikzcd}
\mathsf{Aut}(r) &
\mathsf{Aut}(r \to r') \ar{r}{\pi_{r'}} \ar{l}[swap]{\pi_{r}} &
\mathsf{Aut}(r')
\end{tikzcd}
\end{equation}
for the obvious projections. We now introduce our equivariant generalization of
the ``generalized Reedy categories''
of \cite[Def. 1.1]{BM08}.

\begin{definition}\label{GENRED DEF}
A \textit{generalized Reedy category structure} on a
small category $\mathbb{R}$ consists of
wide subcategories 
$\mathbb{R}^+$, $\mathbb{R}^-$
and a degree function $|\minus| \colon ob(\mathbb{R}) \to \mathbb{N}$ such that:
\begin{itemize}
	\item[(i)] non-invertible maps in $\mathbb{R}^+$ (resp. $\mathbb{R}^-$) raise (lower) degree; isomorphisms preserve degree;
	\item[(ii)] $\mathbb{R}^+ \cap \mathbb{R}^- = \mathsf{Iso}(\mathbb{R})$;
	\item[(iii)] every map $f$ in $\mathbb{R}$ factors as
	$f = f^{+} \circ f^{-}$ with $f^{+} \in \mathbb{R}^+$, $f^{-} \in \mathbb{R}^-$, and this factorization is unique up to isomorphism.
\end{itemize}
Let $\{\mathcal{F}_r\}_{r \in \mathbb{R}}$
be a collection of families of subgroups of the groups $\mathsf{Aut}(r)$.
The collection $\{\mathcal{F}_r\}$ is called 
\textit{Reedy-admissible} if:
\begin{itemize}
	\item[(iv)] for all maps
	$r \twoheadrightarrow r'$ in $\mathbb{R}^-$ one has
	$\pi_{r'}\left( \pi_r^{-1} (H) \right) \in \mathcal{F}_{r'}$
	for all $H \in \mathcal{F}_r$.
\end{itemize}
\end{definition}

We note that condition (iv) above should be thought as of a constraint on the pair 
$(\mathbb{R},\{\mathcal{F}_r\})$.
The original setup of \cite{BM08} then deals with the case
where $\{ \mathcal{F}_r \} =
 \left\{ \left\{ e \right\} \right\}$
is the collection of trivial families. Indeed, our setup recovers
the setup in \cite{BM08}, as follows.

\begin{example}
	When $\{ \mathcal{F}_r \} =
 \left\{ \left\{ e \right\} \right\}$, Reedy-admissibility coincides with axiom (iv) in \cite[Def. 1.1]{BM08},
stating that if $\theta \circ f^{-} = f^{-}$
for some $f^- \in \mathbb{R}^{-}$ and 
$\theta \in \mathsf{Iso}(\mathbb{R})$ then $\theta$ is an identity.
\end{example}

\begin{example}
For any generalized Reedy category $\mathbb{R}$, the collection $\{\mathcal{F}_{\text{all}}\}$
of the families of all subgroups of $\mathsf{Aut}(r)$
is Reedy-admissible.
\end{example}

\begin{example}
	Let $G$ be a group and set $\mathbb{R} = G \times (0 \to 1)$ with $\mathbb{R} = \mathbb{R}^+$. Then any pair 
	$\{\mathcal{F}_0,\mathcal{F}_1\}$
	of families of subgroups of $G$ is Reddy-admissible.
	
	Similarly, set $\mathbb{S} = G \times (0 \leftarrow 1)$
	with $\mathbb{S} = \mathbb{S}^-$. Then a pair
	$\{\mathcal{F}_0,\mathcal{F}_1\}$
	of families of subgroups of $G$ is Reddy-admissible
	iff $\mathcal{F}_0 \supset \mathcal{F}_1$.
\end{example}


\begin{example}\label{GGRAPHREEDY EX}
	Letting $\mathbb{S}$ denote any generalized Reedy category in the sense of \cite[Def. 1.1]{BM08} and $G$ a group,
	we set $\mathbb{R} = G \times \mathbb{S}$
	with $\mathbb{R}^+ = G \times \mathbb{S}^+$ and 
	$\mathbb{R}^- = G \times \mathbb{S}^+$.
	Further, for each $s \in \mathbb{S}$ we write
	$\mathcal{F}_s^{\Gamma}$ for the family of 
	$G$-graph subgroups of $G \times \mathsf{Aut}_{\mathbb{S}}(s)$, i.e., those subgroups 
	$K \leq G \times \mathsf{Aut}_{\mathbb{S}}(s)$ such that $K \cap \mathsf{Aut}_{\mathbb{S}}(s) = \{e\}$.
	
	Reedy admissibility of $\{\mathcal{F}_s^{\Gamma}\}$ follows since for every degeneracy map 
	$s \twoheadrightarrow s'$ in $\mathbb{S}^-$ one has that the homomorphism
	$\pi_s \colon \mathsf{Aut}_{\mathbb{S}}(s \twoheadrightarrow s')
	\to \mathsf{Aut}_{\Omega}(s)$ is injective
	(we note that this is equivalent to axiom (iv) in \cite[Def. 1.1]{BM08} for $\mathbb{S}$).
\end{example}

Our primary example of interest will come by setting
$\mathbb{S} = \Omega^{op}$ in the previous example.
In fact, in this case we will also be interested 
in certain subfamilies
$\{\mathcal{F}_U\}_{U \in \Omega}
\subset \{\mathcal{F}_U^{\Gamma}\}_{U \in \Omega}$.

\begin{example}
	Let $\mathbb{R} = G \times \Omega^{op}$ and let
	$\{\mathcal{F}_U\}_{U \in \Omega}$ be the family of graph subgroups determined by a weak indexing system $\mathcal{F}$.
	Then $\{\mathcal{F}_U\}$ is Reedy-admissible.
	To see this, recall first that each $K \in \mathcal{F}_U$ encodes 
	an $H$-action on $U \in \Omega$ for some $H \leq G$
	so that $G \cdot_H U$ is a $\mathcal{F}$-tree.
	Given a face map $f \colon U' \hookrightarrow U$, 
	the subgroup $\pi^{-1}_U(K)$ is then determined by the largest subgroup $\bar{H}\leq H$ such that 
	$U'$ inherits the $\bar{H}$-action from $U$ along $f$ (so that $f$ becomes a $\bar{H}$-map), 
	so that $\pi_{U'}(\pi^{-1}_U(K))$ encodes the $\bar{H}$-action on $U'$. Thus, we see that Reedy-admissibility is simply the sieve condition for the induced map of $G$-trees
	$G \cdot_{\bar{H}} U' \to G \cdot_H U$.
\end{example}

We now state the main result.
We will assume throughout that $\mathcal{C}$ is a model category such that for any group $G$ and family of subgroups $\mathcal{F}$,
the category $\mathcal{C}^G$ admits the
$\mathcal{F}$-model structure
(for example, this is the case whenever $\C$ is a cofibrantly generated cellular model category in the sense of \cite{Ste16}).


\begin{theorem}\label{REEDYADM THM}
Let $\mathbb{R}$ be generalized Reedy and 
$\{\mathcal{F}_r\}_{r \in \mathbb{R}}$ a Reedy-admissible collection of families. 
Then there is a \textbf{$\{\mathcal{F}_r\}$-Reedy model structure} on
$\mathcal{C}^{\mathbb{R}}$ such that a map $A \to B$ is
\begin{itemize}
  \item a (trivial) cofibration if $A_r \underset{L_r A}{\coprod}L_r B \to B_r$ is a (trivial) $\mathcal{F}_r$-cofibration in $\mathcal{C}^{\mathsf{Aut}(r)}$, $\forall r \in \mathbb{R}$;
	\item a weak equivalence if $A_r \to B_r$ is a $\mathcal{F}_r$-weak equivalence in $\mathcal{C}^{\mathsf{Aut}(r)}$, $\forall r \in \mathbb{R}$;
	\item a (trivial) fibration if $A_r \to B_r \underset{M_r B}{\times }M_r A $ is a (trivial) $\mathcal{F}_r$-fibration in $\mathcal{C}^{\mathsf{Aut}(r)}$, $\forall r \in \mathbb{R}$.
\end{itemize}
\end{theorem}

The proof of this result is given at the end of the section after establishing some routine generalizations of the key lemmas in \cite{BM08}
(indeed, the true novelty in this appendix is the Reedy-admissibility condition in part (iv) of Definition \ref{GENRED DEF}).

We first recall the following, cf. \cite[Props. 6.5 and 6.6]{BP17}
(we note that \cite[Prop. 6.6]{BP17} can be proven in terms of fibrations, and thus does not depend on special assumptions on $\C$).
\begin{proposition}
Let $\phi \colon G \to \bar{G}$ be a homomorphism and
$\mathcal{F}$, $\bar{\mathcal{F}}$ families of subgroups of
$G, \bar{G}$. Then the leftmost (resp. rightmost) adjunction below
is a Quillen adjunction 
\[
	\bar{G} \cdot_G (\minus)
	\colon \C^G_{\mathcal{F}}
		\rightleftarrows
	\C^{\bar{G}}_{\bar{\mathcal{F}}} \colon
	\mathsf{res}^{\bar{G}}_G
\qquad
	\mathsf{res}^{\bar{G}}_G
	\colon	\C^{\bar{G}}_{\bar{\mathcal{F}}}
		\rightleftarrows
	\C^G_{\mathcal{F}} \colon
	\mathsf{Hom}_G(\bar{G},\minus)
\]
provided that for $H \in \mathcal{F}$ it is
$\phi(H) \in \bar{\mathcal{F}}$
(resp. for $\bar{H} \in \bar{\mathcal{F}}$ it is
$\phi^{-1}(H) \in \mathcal{F}$).
\end{proposition}



\begin{corollary}\label{RESGEN COR}
For any homomorphism $\phi \colon G \to \bar{G}$, the functor
$\mathsf{res}^{\bar{G}}_G \colon 
\C^{\bar{G}} \to \C^G$
preserves all four classes of genuine cofibrations, trivial cofibrations, fibrations and trivial fibrations.
\end{corollary}

The following formalizes an argument implicit in the proof of \cite[Lemma 5.2]{BM08}).

\begin{definition}
Consider a commutative diagram
\begin{equation}\label{BLA EQ}
	\begin{tikzcd}
		A \ar{r} \ar{d} & X \ar{d}
	\\
		B \ar{r} \ar[dashed]{ru} & Y
	\end{tikzcd}
\end{equation}
in $\C^{\mathbb{R}}$. A collection of maps 
$f_s \colon B_s \to X_s$ for $|s|\leq n$ 
that induce a lift of the restriction of (\ref{BLA EQ})
 to $\C^{\mathbb{R}_{\leq n}}$ will be called a 
\textit{$n$-partial lift}. 
\end{definition}


\begin{lemma}\label{BLALIFT LEM}
	Let $\C$ be any bicomplete category, and consider a commutative diagram as in (\ref{BLA EQ}). Then any $(n-1)$-partial lift uniquely induces commutative diagrams
\begin{equation}\label{BLALIFT EQ}
	\begin{tikzcd}
		A_r \amalg_{L_r A} L_r B \ar{r} \ar{d} & X_r \ar{d}
	\\
		B_r \ar{r} \ar[dashed]{ru} & Y_r \times_{M_r Y} M_r X
	\end{tikzcd}
\end{equation}
in $\mathcal{C}^{\mathsf{Aut}(r)}$
for each $r$ such that $|r|=n$. Furthermore, extensions of the 
$(n-1)$-partial lift to a $n$-partial lift are in bijection with choices of $\mathsf{Aut}(r)$-equivariant lifts of the diagrams (\ref{BLALIFT EQ}) for $r$ ranging over representatives of the isomorphism classes of $r$ with $|r|=n$.
\end{lemma}

In the next result, by $\{\mathcal{F}_r\}$-cofibration/trivial cofibration/fibration/trivial fibration 
we mean a map as described in 
Theorem \ref{REEDYADM THM}, regardless of whether such a model structure exists.

\begin{corollary}\label{BLALIFT COR}
Let $\mathbb{R}$ be generalized Reedy and 
$\{\mathcal{F}_r\}$ an arbitrary family of subgroups of $\mathsf{Aut}(r)$, $r \in \mathbb{R}$.
Then a map in $\mathcal{C}^{\mathbb{R}}$ 
is a $\{\mathcal{F}_r\}$-cofibration (resp. trivial cofibration) iff it has the left lifting property 
with respect to all 
$\{\mathcal{F}_r\}$-trivial fibrations (resp. fibrations),
and vice-versa for the right lifting property.
\end{corollary}

\begin{lemma}\label{GINJ LEM}
Let $\mathbb{S}$ be a generalized Reedy with $\mathbb{S}=\mathbb{S}^+$, $K$ a group, and $\pi \colon \mathbb{S} \to K$
a functor.

Then if a map $A \to B$ in $\C^{\mathbb{S}}$ is such that for all 
$s \in \mathbb{S}$
the maps 
$
  A_s \amalg_{L_s A} L_s B \to B_s
$	
are (resp. trivial) $\mathsf{Aut}(s)$-cofibrations one has that
$\mathsf{Lan}_{\pi\colon \mathbb{S} \to K}(A \to B)$
is a (trivial) $K$-cofibration.
\end{lemma}

\begin{proof}
By adjunction, one needs only show that for any 
$K$-fibration $X \to Y$ in $\mathcal{C}^K$,
the map $\pi^{\**}(X \to Y)$
has the right lifting property against all maps $A \to B$ in $\C^{\mathbb{S}}$ as in the statement.
By Corollary \ref{BLALIFT COR}, it thus suffices to check
that the maps
\[
	(\pi^{\**} X)_s \to 
	(\pi^{\**} Y)_s \times_{M_s \pi^{\**} Y} M_s \pi^{\**} X
\]
are $\mathsf{Aut}(s)$-fibrations. But since $M_s Z = \**$ 
(recall $\mathbb{S}=\mathbb{S}^+$)
this map is just $X \to Y$ with the $\mathsf{Aut}(s)$-action induced by
$\pi \colon \mathsf{Aut}(s) \to K$, hence 
Corollary \ref{RESGEN COR} finishes the proof.
\end{proof}


\begin{lemma}\label{GINJMIN LEM}
Let $\mathbb{S}$ be a generalized Reedy with $\mathbb{S}=\mathbb{S}^-$, $K$ a group, and $\pi \colon \mathbb{S} \to K$
a functor.

Then if a map $X \to Y$ in $\C^{\mathbb{S}}$ is such that for all 
$s \in \mathbb{S}$
the maps 
$
	X_s \to Y_s \times_{M_s Y} M_s X
$	
are (resp. trivial) $\mathsf{Aut}(s)$-fibrations one has that
$\mathsf{Ran}_{\pi\colon \mathbb{S} \to K}(A \to B)$
is a (trivial) $K$-fibration.
\end{lemma}

\begin{proof}
This follows dually to the previous proof.
\end{proof}

\begin{remark}
Lemmas \ref{GINJ LEM} and \ref{GINJMIN LEM} generalize key parts of the proofs of \cite[Lemmas 5.3 and 5.5]{BM08}.  
The duality of their proofs reflects the duality in 
Corollary \ref{RESGEN COR}.
\end{remark}

\begin{remark}
	Lemma \ref{GINJ LEM} will be applied when
	$K \leq \mathsf{Aut}_{\mathbb{R}}(r)$ and
	$\mathbb{S} = K \ltimes \mathbb{R}^+(r)$ for $\mathbb{R}$ a given generalized Reedy category and $r \in \mathbb{R}$.
	Similarly, Lemma \ref{GINJMIN LEM} will be applied when
	$\mathbb{S} = K \ltimes \mathbb{R}^-(r)$.
	It is straightforward to check that in the $\mathbb{R}^+$ (resp. $\mathbb{R}^-$) case
	maps in $\mathbb{S}$ can be identified with squares as on the left (right)
	\begin{equation}
	\begin{tikzcd}
		r' \ar{r}{+} \ar{d}[swap]{+} & r \ar{d}{\simeq}
	& &
		r \ar{r}{-} \ar{d}[swap]{\simeq} & r' \ar{d}{-}
	\\
		r'' \ar{r}[swap]{+} & r
	& &
		r \ar{r}[swap]{-} & r''
	\end{tikzcd}
\end{equation}
such that the maps labelled $+$ are in $\mathbb{R}^+$,
maps labelled $-$ are in $\mathbb{R}^-$,
the horizontal maps are non-invertible, and the maps labeled $\simeq$ are automorphisms in $K$. 

In particular, there is thus a \textit{domain} (resp. \textit{target}) functor
$d \colon \mathbb{S} \to \mathbb{R}$ 
($t \colon \mathbb{S} \to \mathbb{R}$), and our interest is in maps  
$d^{\**}A \to d^{\**} B$
($t^{\**}A \to t^{\**} B$) in $\mathcal{C}^{\mathbb{S}}$
induced from maps
$A \to B$ in $\C^{\mathbb{R}}$ so that
\[\mathsf{Lan}_{\pi} d^{\**} (A \to B) = 
(L_r A \to L_r B)
	\qquad
\mathsf{Ran}_{\pi} t^{\**} (A \to B) = 
(M_r A \to M_r B)
\]
\end{remark}

We are now in a position to prove the following, which are the essence of Theorem \ref{REEDYADM THM}.

\begin{lemma}\label{REEDYTRCOF LEM}
Let $\mathbb{R}$ be generalized Reedy and 
$\{\mathcal{F}_r\}_{r \in \mathbb{R}}$ a Reedy-admissible family.

Suppose $A \to B$ be a $\{\mathcal{F}_r\}$-Reedy cofibration. Then the maps $A_r \to B_r$ are all $\{\mathcal{F}_r\}$-weak equivalences iff so are the maps $A_r \amalg_{L_r A} L_r B \to B_r$.
\end{lemma}

\begin{proof}
It suffices to check by induction on $n$ that the analogous claim with the restriction $|r|\leq n$ also holds. The $n=0$ case is obvious. Otherwise, letting $r$ range over representatives of the isomorphism classes of $r$ with $|r|=n$,
it suffices to check that for each $H \in \mathcal{F}_r$
the map
$A_r \to B_r$ is a $H$-genuine weak equivalence iff 
so is $A_r \amalg_{L_r A} L_r B \to B_r$.

One now applies Lemma \ref{GINJ LEM} with 
$K = H$ and 
$\mathbb{S} = H \ltimes \mathbb{R}^+(r)$
to the map $d^{\**}A \to d^{\**}B$. Note that $\mathcal{F}$-trivial cofibrations are always genuine trivial cofibrations, for any family, so that the trivial cofibrancy requirements are immediate from Corollary \ref{RESGEN COR}. 
It thus follows that the maps labelled $\sim$
\[
\begin{tikzcd}[row sep=10]
   L_r A \ar{r}{\sim} \ar[d]   & 
   L_r B \ar[d] & 
\\
   A_r \ar{r}[swap]{\sim}& L_T B \amalg_{L_T A} A_T \ar{r} &
   B_r
\end{tikzcd}
\]
are $H$-genuine trivial cofibrations, finishing the proof.
\end{proof}

\begin{lemma}\label{REEDYTRFIB LEM}
Let $\mathbb{R}$ be generalized Reedy and 
$\{\mathcal{F}_r\}_{r \in \mathbb{R}}$ a Reedy-admissible family.

Let $X \to Y$ be a $\{\mathcal{F}_r\}$-Reedy fibration. Then the maps $X_r \to Y_r$ are all $\{\mathcal{F}_r\}$-weak equivalences iff so are the maps $X_r \to Y_r \times_{M_r Y} M_r X$.
\end{lemma}

\begin{proof}
One repeats the same induction argument on $|r|$.
In the induction step, it suffices to verify that, for each $r$ with $|r|=n$ and $H \in \mathcal{F}_r$, the map
$X_r \to Y_r$ is a $H$-genuine weak equivalence iff 
so is $X_r \to Y_r \times_{M_r Y} M_r X$.

One now applies Lemma \ref{GINJMIN LEM} with 
$K = H$ and 
$\mathbb{S} = H \ltimes \mathbb{R}^-(r)$
to the map $t^{\**}A \to t^{\**}B$. 
Note that for each $(r \twoheadrightarrow r') \in \mathbb{S}$ one has $\mathsf{Aut}_{\mathbb{S}}(r \to r') = \pi_r^{-1}(H)$
(where $\pi_r$ is as in (\ref{PIDEFR EQ})), so that the trivial fibrancy requirement in Lemma \ref{GINJMIN LEM} follows from 
$\{\mathcal{F}_r\}$ being Reedy-admissible.
It follows that the maps labelled $\sim$
\[
\begin{tikzcd}[row sep=10]
	X_r \ar{r}&
	Y_r \times_{M_r Y} M_r X \ar[d] \ar{r}{\sim} & 
	Y_r \ar[d]
\\
	&
	M_r X \ar{r}[swap]{\sim}&
	M_r Y
\end{tikzcd}
\]
are $H$-genuine trivial fibrations, finishing the proof.
\end{proof}

\begin{remark}
The proofs of Lemmas \ref{REEDYTRCOF LEM} and \ref{REEDYTRFIB LEM}
are similar, but not dual, since
Lemma \ref{REEDYTRFIB LEM} uses Reedy-admissibility 
while Lemma \ref{REEDYTRCOF LEM} does not.
This reflects the difference in the proofs of 
\cite[Lemmas 5.3 and 5.5]{BM08} as discussed in 
\cite[Remark 5.6]{BM08}, albeit with a caveat.

Setting $K=\{e\}$ in Lemma \ref{GINJ LEM} yields that
$\mathsf{lim}_{\mathbb{S}} (A \to B)$ is a cofibration provided that $A \to B$ is a genuine Reedy cofibration, i.e. a Reedy cofibration for $\{\mathcal{F}_{\text{all}}\}$ the families of all subgroups. 
On the other hand, the proof of \cite[Lemma 5.3]{BM08} argues that 
$\mathsf{lim}_{\mathbb{S}} (A \to B)$ is a cofibration provided that $A \to B$ is a projective Reedy cofibration, i.e. a Reedy cofibration for $\{\{e\}\}$ the trivial families 
(note that all projective cofibrations are genuine cofibrations, so that our claim is more general).
Since the cofibration half of the projective analogue of Corollary \ref{RESGEN COR} only holds if $\phi$ is a monormorphism, the argument in the proof of \cite[Lemma 5.3]{BM08} also includes an injectivity check that is not needed for our proof of Lemma \ref{REEDYTRCOF LEM}.
\end{remark}


\begin{proof}[proof of Theorem \ref{REEDYADM THM}]
Lemmas \ref{REEDYTRCOF LEM} and \ref{REEDYTRFIB LEM} say that the characterizations of trivial cofibrations (resp. trivial fibrations) in the statement of Theorem \ref{REEDYADM THM} are correct, i.e. that they describe the maps that are both cofibrations (resp. fibrations) and weak equivalences.	

	We refer to the model category axioms in \cite[Def. 1.1.3]{Hov98}. 	
	Both 2-out-of-3 and the retract axioms are immediate
(recall that retracts commute with limits/colimits).	
	The lifting axiom follows from Corollary \ref{BLALIFT COR}
	while the task of building factorizations $X \to A \to Y$ of a given map $X \to Y$ follows by a similar standard argument by iteratively factorizing the maps
\[
	X_r \amalg_{L_r X} L_r A \to Y_r \times_{M_r Y} M_r A
\]
in $\mathcal{C}^{\mathsf{Aut}(r)}$, 
thus building both $A$ and the factorization inductively (see, e.g., the proof of \cite[Thm. 1.6]{BM08}).
\end{proof}






\bibliography{biblio}{}



\bibliographystyle{abbrv}



\end{document}